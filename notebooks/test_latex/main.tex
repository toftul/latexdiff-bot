\documentclass[journal=chreay,manuscript=review]{achemso}
\usepackage[]{achemso}

\setkeys{acs}{maxauthors = 30}
%\usepackage[maxauthors = 50]{achemso}

%%%%%%%%%%%%%%%%%%%%%%%%%%%%%%%%%%%%%%%%%%%%%%%%%%%%%%%%%%%%%%%%%%%%%
%% Place any additional packages needed here.  Only include packages
%% which are essential, to avoid problems later. Do NOT use any
%% packages which require e-TeX (for example etoolbox): the e-TeX
%% extensions are not currently available on the ACS conversion
%% servers.
%%%%%%%%%%%%%%%%%%%%%%%%%%%%%%%%%%%%%%%%%%%%%%%%%%%%%%%%%%%%%%%%%%%%%
\usepackage[version=3]{mhchem} % Formula subscripts using \ce{}
\usepackage{xcolor}
\usepackage[normalem]{ulem}
%\usepackage[colorlinks=true,bookmarks=false,citecolor=blue,urlcolor=blue]{hyperref} 
\usepackage[table,xcdraw]{xcolor}


%%%%%%%%%%%%%%%%%%%%%%%%%%%%%%%%%%%%%%%%%%%%%%%%%%%%%%%%%%%%%%%%%%%%%
%% If issues arise when submitting your manuscript, you may want to
%% un-comment the next line.  This provides information on the
%% version of every file you have used.
%%%%%%%%%%%%%%%%%%%%%%%%%%%%%%%%%%%%%%%%%%%%%%%%%%%%%%%%%%%%%%%%%%%%%
%%\listfiles

%%%%%%%%%%%%%%%%%%%%%%%%%%%%%%%%%%%%%%%%%%%%%%%%%%%%%%%%%%%%%%%%%%%%%
%% Place any additional macros here.  Please use \newcommand* where
%% possible, and avoid layout-changing macros (which are not used
%% when typesetting).
%%%%%%%%%%%%%%%%%%%%%%%%%%%%%%%%%%%%%%%%%%%%%%%%%%%%%%%%%%%%%%%%%%%%%
\newcommand*\mycommand[1]{\texttt{\emph{#1}}}
\newcommand{\red}{\textcolor{red}} 
\newcommand{\blue}{\textcolor{blue}}
\newcommand{\specialcell}[2][c]{%
  \begin{tabular}[#1]{@{}c@{}}#2\end{tabular}}
%\newcommand{\specialcellthree}[3][c]{%
%  \begin{tabular}[#1]{@{}c@{}}#2\end{tabular}}

%\usepackage[colorlinks=false,bookmarks=false,citecolor=blue,urlcolor=blue]{hyperref} 

%% Chem. Rev. toc image is 5 by 5 cm
\makeatletter
\setlength\acs@tocentry@height{5cm}
\setlength\acs@tocentry@width{5cm}
\makeatother


%%%%%%%%%%%%%%%%%%%%%%%%%%%%%%%%%%%%%%%%%%%%%%%%%%%%%%%%%%%%%%%%%%%%%
%% Meta-data block
%% ---------------
%% Each author should be given as a separate \author command.
%%
%% Corresponding authors should have an e-mail given after the author
%% name as an \email command. Phone and fax numbers can be given
%% using \phone and \fax, respectively; this information is optional.
%%
%% The affiliation of authors is given after the authors; each
%% \affiliation command applies to all preceding authors not already
%% assigned an affiliation.
%%
%% The affiliation takes an option argument for the short name.  This
%% will typically be something like "University of Somewhere".
%%
%% The \altaffiliation macro should be used for new address, etc.
%% On the other hand, \alsoaffiliation is used on a per author basis
%% when authors are associated with multiple institutions.
%%%%%%%%%%%%%%%%%%%%%%%%%%%%%%%%%%%%%%%%%%%%%%%%%%%%%%%%%%%%%%%%%%%%%
\author{Pavel Tonkaev}
\affiliation{Nonlinear Physics Center, Research School of Physics, Australian National University, Canberra, ACT 2601, Australia}
\alsoaffiliation{School of Physics and Engineering, ITMO University, St. Petersburg 197101, Russia}

\author{Ivan S. Sinev}
\affiliation{School of Physics and Engineering, ITMO University, St. Petersburg 197101, Russia}

\author{Mikhail V. Rybin}
\affiliation{School of Physics and Engineering, ITMO University, St. Petersburg 197101, Russia}
\alsoaffiliation{Ioffe Institute, Russian Academy of Science, St.~Petersburg 194021, Russia}

%\altaffiliation{A shared footnote}

\author{Sergey V. Makarov}
\affiliation{School of Physics and Engineering, ITMO University, St. Petersburg 197101,  Russia}

\author{Yuri Kivshar}
\affiliation{School of Physics and Engineering, ITMO University, St. Petersburg 197101, Russia}
\alsoaffiliation{Nonlinear Physics Center, Research School of Physics, Australian National University, Canberra, ACT 2601, Australia}
\email{yuri.kivshar@anu.edu.au}



%%%%%%%%%%%%%%%%%%%%%%%%%%%%%%%%%%%%%%%%%%%%%%%%%%%%%%%%%%%%%%%%%%%%%
%% The document title should be given as usual. Some journals require
%% a running title from the author: this should be supplied as an
%% optional argument to \title.
%%%%%%%%%%%%%%%%%%%%%%%%%%%%%%%%%%%%%%%%%%%%%%%%%%%%%%%%%%%%%%%%%%%%%
\title[]
  {Multifunctional and Transformative Metaphotonics with Emerging Materials}

%%%%%%%%%%%%%%%%%%%%%%%%%%%%%%%%%%%%%%%%%%%%%%%%%%%%%%%%%%%%%%%%%%%%%
%% Some journals require a list of abbreviations or keywords to be
%% supplied. These should be set up here, and will be printed after
%% the title and author information, if needed.
%%%%%%%%%%%%%%%%%%%%%%%%%%%%%%%%%%%%%%%%%%%%%%%%%%%%%%%%%%%%%%%%%%%%%
\abbreviations{IR,NMR,UV}
\keywords{American Chemical Society, \LaTeX}
\DeclareUnicodeCharacter{2212}{-}

%%%%%%%%%%%%%%%%%%%%%%%%%%%%%%%%%%%%%%%%%%%%%%%%%%%%%%%%%%%%%%%%%%%%%
%% The manuscript does not need to include \maketitle, which is
%% executed automatically.
%%%%%%%%%%%%%%%%%%%%%%%%%%%%%%%%%%%%%%%%%%%%%%%%%%%%%%%%%%%%%%%%%%%%%
\begin{document}

%%%%%%%%%%%%%%%%%%%%%%%%%%%%%%%%%%%%%%%%%%%%%%%%%%%%%%%%%%%%%%%%%%%%%
%% The "tocentry" environment can be used to create an entry for the
%% graphical table of contents. It is given here as some journals
%% require that it is printed as part of the abstract page. It will
%% be automatically moved as appropriate.
%%%%%%%%%%%%%%%%%%%%%%%%%%%%%%%%%%%%%%%%%%%%%%%%%%%%%%%%%%%%%%%%%%%%%
\begin{tocentry}

\centering 
\includegraphics[width=5cm]{Figs/Fig_TOC.png}



\end{tocentry}

%%%%%%%%%%%%%%%%%%%%%%%%%%%%%%%%%%%%%%%%%%%%%%%%%%%%%%%%%%%%%%%%%%%%%
%% The abstract environment will automatically gobble the contents
%% if an abstract is not used by the target journal.
%%%%%%%%%%%%%%%%%%%%%%%%%%%%%%%%%%%%%%%%%%%%%%%%%%%%%%%%%%%%%%%%%%%%%
\begin{abstract}
Future technologies underpinning multifunctional  physical and chemical systems and compact biological sensors will rely on densely packed transformative and tunable circuitry employing {\it nanophotonics}. For many years, plasmonics was considered as the only available platform for subwavelength optics, but the recently emerged field of resonant metaphotonics may provide a versatile  practical platform for nanoscale science by employing resonances in high-index dielectric nanoparticles and metasurfaces. Here, we discuss the recently emerged field of metaphotonics and describe its connection to material science and chemistry.  For tunabilty, metaphotonics employs a variety of the recently highlighted materials such as polymers, perovskites, transition metal dichalcogenides, and phase-changed materials. This allows to achieve diverse functionalities of metasystems and metasurfaces for efficient spatial and temporal control of light by employing multipolar resonances and the physics of bound states in the continuum. We anticipate expanding applications of these concepts in nanolasers, tunable metadevices, metachemistry, as well as a design of a new generation of chemical and biological ultracompact sensing devices. 
\end{abstract}

\newpage

\tableofcontents

\newpage

%%%%%%%%%%%%%%%%%%%%%%%%%%%%%%%%%%%%%%%%%%%%%%%%%%%%%%%%%%%%%%%%%%%%%
%% Start the main part of the manuscript here.
%%%%%%%%%%%%%%%%%%%%%%%%%%%%%%%%%%%%%%%%%%%%%%%%%%%%%%%%%%%%%%%%%%%%%
\section{Introduction}

{\bf From metamaterials to metaphotonics}. The concept of {\it metamaterials} originates from earlier ideas to create artificial media with negative refraction, which can be observed for light bending in the direction opposite to the prediction of the classical Snell’s law when light enters another media. This negative refraction can be achieved with composite media having properties not found in nature, and when the material is composed of subwavelength elements being smaller than the wavelengths of the propagating electromagnetic waves~\cite{xiang_2011}. The initial experimental demonstration of metamaterials was based on wire rods and split-ring resonators \cite{smith_2004} which create a periodic lattice of subwavelength elements that can be described by the effective material parameters for propagating microwaves.

In metamaterials, the response of the subwavelength constituent elements contributes to the averaged parameters such as electrical permittivity and magnetic permeability. Exotic properties of light propagating in metamaterials attracted a lot of attention in both physics and chemistry in the past years. This field involves both propagation and localization of light on the spatial scales much smaller than the optical wavelength, and largely surpassing the classical limits imposed by diffraction. Looking deeper into the properties of resonant composite media not described by averaged parameters, it brings us to the emerging field of {\it metaphotonics} that studies both electric and magnetic response and resonances at the level of constituent elements, meta-atoms (Fig.~\ref{fig:Concepts}). Metaphotonics combines the concepts of metamaterials and nanophotonics, and it offers a unique platform for achieving an unusual electromagnetic response on a scale much smaller than the wavelength of light, although practical applications are strongly hampered by device fabrication complexity. Importantly, the concepts of metaphotonics can be extended to quantum technologies, demonstrating the feasibility of metadevices~\cite{metadevices} for both manipulation and measurement of quantum states.

Traditionally, an efficient control of light at the subwavelength scales was realized solely with metallic surfaces or interfaces that can support  hybrid modes originating from coupling of electromagnetic waves with with free electrons, the so-called surface plasmon polaritons (or plasmons) and localized plasmonic resonances, which enable subwavelength-scale light confinement~\cite{plasmonics} (Fig.~\ref{fig:Concepts}). In the recent years, we observe the emergence of a new field of resonant dielectric metaphotonics~\cite{kuznetsov2016optically} that provides a novel platform for the subwavelength optics. All-dielectric resonant nanophotonics employs subwavelength electric and magnetic Mie resonances of high-index dielectric nanoparticles as elementary units of nanophotonic structures. Sometimes, this novel direction is termed as {\it Mie-tronics}~\cite{Mie_won}. 

Many advantages of resonant dielectric structures in comparison with their plasmonic analogues include low losses and the additional enhancement of the magnetic field, and they can lead to advanced effects originating from strong coupling and interference of several optically resonant modes~\cite{Koshelev_2021}. The field of dielectric resonant metaphotonics has attracted a lot of attention from researchers of diverse fields, and it can lead to many advanced applications in chemical and biological sensing, imaging, and quantum technologies. 

Many effects involving the subwavelength localization of light are based on the idea to confine light in an optical resonator~\cite{vahala_2003}. A typical optical resonator traps light with the help of “mirrors", and it is based on the physics of total internal reflection or employs the multiple back-scatterings from periodic structures such as photonic crystals. Efficiency of light trapping in this type of optical system can be characterized by the quality factor ($Q$ factor) being weakly dependent on the resonator geometry. Importantly, strong confinement of light and high $Q$-factor resonances are also available with Mie-resonant high-index dielectric structures~\cite{Koshelev_2021}. More specifically, they employ optical resonances and nonradiating current configurations (such as anapoles and bound states in the continuum~\cite{BIC_2019}) that can induce the large field enhancements.

Unlike plasmonic structures, the electromagnetic fields penetrate deep inside dielectric nanostructures, so the field enhancement in dielectrics should be designed on the basis of different physical principles, in comparison with metallic structures. Such principles rely on the geometric resonances and interferences between different modes, so the Mie resonances play an important role in all-dielectric metaphotonics~\cite{kuznetsov2016optically,staude2017metamaterial,kruk2017functional}, and they can be employed for the control of light below the free-space diffraction limit. The Mie resonances are usually associated with a large concentration of the electromagnetic field inside particles~\cite{kapitanova2017giant} suggesting at least tenfold enhancement of many optical effects observed with dielectric nanostructures being critical for boosting many effects in nonlinear nanophotonics~\cite{shcherbakov2014enhanced,makarov2015tuning,shcherbakov2017ultrafast,cambiasso2017bridging,zhang2018lighting,marino2019zero,zograf2020stimulated,zograf2022high}.

{\bf Metasystems and metasurfaces}.  As mentioned above, metamaterials appeared as artificial structures composed of subwavelength elements suggested initially to realize the effect of negative refraction. 
Later, metamaterials were developed as a new platform for engineering electromagnetic space with the help of transformation optics. Current research on metamaterials evolves the study of {\it metasystems} and {\it metasurfaces} as components of the so-called {\it metadevices} defined as optical devices having unique functionalities achieved by patterning of functional matter on the subwavelength scale~\cite{metadevices}. It is expected that the future technologies based on dense photonic integration will employ the material concepts and their building blocks such as metasystems and metasurfaces. 

\begin{figure}[ht!]
    \centering
    \includegraphics[width=0.75\linewidth]{Figs/Intro-concepts-v4.png}
    \caption{Basic concepts of metaphotonics and their realization with metasurfaces. Plasmonic and dielectric (Mie) resonators are employed as elementary units - meta-atoms. Hybridization of the optical modes of resonators happens when they are brought close together not unlike the hybridization of electronic states during the formation of chemical compounds. Bottom: various metaphotonics functionalities that emerge when meta-atoms are arranged in lattices (metasystems or metasurfaces). Spectral shaping involves engineering the transmission, reflection and absorption spectra. Purcell effect refers to modification of the spontaneous emission rate due to artificially increased density of optical states. Light routing allows to control the propagation direction of light beams. Integration of metaphotonic devices with light-emitting materials allows to control the efficiency and directivity of light emission. Giant field enhancements can be achieved via engineering of high quality factor resonances. Polarization conversion provided by metasurfaces allow for additional information encoding and holographic applications. }
    \label{fig:Concepts}
\end{figure}

{\it Metasystems} usually involved subwavelength resonant elements such as nanoantennas, and they support different types of local resonances operating as building blocks for generating, manipulating, and modulating light. By exciting both electric and magnetic multipolar Mie resonances, one can modify the far-field radiation of metasystems and also localize the electromagnetic energy in open resonators by employing the physics of bound states in the continuum~\cite{BIC_2019}.

{\it Metasurfaces} are created by artificial subwavelength elements with small thickness which provides novel capability to manipulate electromagnetic waves~\cite{kildishev_2013}. Well before the exploration of metasurfaces, tailoring the light scattering with planar optical structures has been majorly pursued with diffractive optical elements~\cite{lalanne_2017}. However, the concept of metasurfaces provides much broader and deeper insights and useful tools for complete control of light. Metasurfaces are characterized by reduced dimensionality, and usually they consist of arrays of optical resonators with spatially varying geometric parameters and subwavelength separation. In contrast to conventional optical components that achieve wavefront engineering by phase accumulation through light propagation in a medium, metasurface provides new degrees of freedom to control the phase, amplitude, and polarization of light waves with subwavelength resolution, as well as to accomplish wavefront shaping within a distance much less than the wavelength of light\cite{jung2021metasurface}. The outstanding optical properties of dielectric metasurfaces drive the development of ultra-thin optical elements and devices, whether showing novel optical phenomena or new functionalities outperforming their traditional bulky counterparts~\cite{levy_2020}. Metasurfaces consist of carefully arranged “unit cells” or “meta-atoms” with subwavelength structures. The optical response (phase, amplitude, and polarisation) of the meta-atom changes with its geometry (height, width, material, etc). The meta-atoms are arranged into arrays to provide specific variations of parameters, depending on required functionalities. Meta-atoms can operate as subwavelength resonators supporting multipolar Mie resonances~\cite{kruk2017functional}, or they can contribute to averaged parameters like metamaterials~\cite{tsai_2017}.

The concept of optical metasurfaces was applied to demonstrate many exotic optical phenomena and various useful planar optical devices. Many of these metasurface-based applications are potentially very promising alternatives to replace conventional optical elements and devices, as they largely benefit from ultra-thin, lightweight, and ultracompact properties, provide the possibility of overcoming several limitations suffered by their traditional counterparts, and can demonstrate versatile novel functionalities. Metasurfaces were suggested for an efficient control of light-matter interaction with subwavelength resonant structures, and they have been explored widely in the recent years for creating transformational flat-optics devices.

As discussed above, dielectric metastructures and metasurfaces offer critical advantages over their plasmonic counterparts and provide means for efficient control over optical properties such as amplitude, phase, polarization, chirality, and anisotropy~\cite{cwqiu_2021}. The control of all such parameters require a careful optimisation depending on the problems where the metasurfaces are used. Many such properties are driven by local electromagnetic resonances such as Mie-type modes~\cite{zhao2009mie,kuznetsov2016optically}, bound states in the continuum~\cite{hsu2016bound,BIC_2019}, Fano resonances~\cite{miroshnichenko2010fano,limonov2017fano}, and anapole resonances~\cite{miroshnichenko2015nonradiating,yang2019nonradiating}. In turn, these phenomena underpin diverse functionalities provided by metaphotonic structures as summarized in the right panel of Fig.~\ref{fig:Concepts}. These include light routing \textit{via} engineered electromagnetic phase gradients or diffraction; spectral shaping through management of resonances and their mutual interference; Purcell effect (acceleration of light emission dynamics~\cite{purcell1995spontaneous}) enabled by designing the local density of optical states; local electromagnetic field enhancement that facilitates nonlinear processes and benefits sensing applications; and polarization conversion for information encoding.

Importantly, recent advances in nanofabrication technologies bring low-cost, large-area, as well as mass productive approaches and capabilities for the development of various types of metasystems and metasurfaces, and those methods are gradually becoming mature~\cite{clarke2018large,lee2018metasurface, das2019self, li2020large,yoon2020single,zhizhchenko2020light,jung2021three}. It is expected that flat-optics components based on dielectric metasurfaces will appear in our daily life very soon bringing complexity of optical components and novel functionalities~\cite{capasso}. 

\begin{figure}[t!]
    \centering
    \includegraphics[width=0.8\linewidth]{Figs/Intro-applications.png}
    \caption{A summary of the basic metaphotonics functionalities (grey sections below) and emerging materials (center) generating various applications (above). Each material is encoded with a specific color that matches the chart of potential applications.}
    \label{fig:Intro}
\end{figure}

{\bf Metaphotonics and chemistry}. Metaphotonics as an emerging branch of interdisciplinary research is inspired by the progress in condensed matter physics and chemistry. Indeed, many concepts that appeared in the 20th century demanded the development of unusual properties of materials delivered by chemistry. A progress in nanotechnology and artificial materials was accompanied by the reduction of the fingerprint of optical structures down to the scales of several hundred nanometers, which resulted in the birth of metaphotonics employing the lattices of resonant meta-atoms forming artificial materials. Still, many of the approaches of metaphotonics can be directly linked to the concepts that are very well-known in chemistry.

As a primary example, $p-$ and $d-$ atomic orbitals are converted into dipole and quadrupole modes of meta-atoms in metaphotonics. The $s$-type orbitals, however, have no analogue in photonics because of the vectorial nature of electromagnetic waves and symmetry considerations, so that the lowest in frequency (energy) $p$-type dipole modes of meta-atoms are usually sufficient to describe the basic properties of metaphotonics structures. Furthermore, the ``metaphotonic counterpart'' of the chemical bonds is the  optical mode hybridization between neighboring meta-atoms. Though the hybridized state does not respond for mechanical stability as chemical bonding does, the mode splitting takes place leading to the formation of a meta-molecule, or, eventually, a meta-crystal. The degree of the interparticle mode hybridization (as it is the case for bonding in chemistry) plays a crucial part in the properties of metastructures. 

Strong link between chemistry and metaphotonics is manifested further in the fact the designs of metaphotonics structures rely heavily on the ``classical'' chemistry of constitutive materials, since it defines the functional properties and even the possible geometry of meta-atoms. In turn, metaphotonics devices open many opportunities for the diagnostics of chemical composition because they strongly enhance light-matter interaction. In particular, it is very promising to exploit high-$Q$ optical modes supported by metasurfaces that enable precise sensing of molecules by their vibrational modes in the infrared frequency range. Also, the concentration of the electromagnetic field into localized hot spots makes it possible to simulate chemical reactions in selected spatial regions, in particular, by photo-initiated polymerization of photoresist. Thus, metaphotonics is connected to chemistry at different logical levels including both concepts and ideas for new applications and designs.


\begin{figure}[b!]
    \centering
    \includegraphics[width=\linewidth]{Figs/Intro-tunable.png}
    \caption{Emerging materials employed for metaphotonics and discussed in this review. Materials are classified from the viewpoint of their tunability under different external stimuli. }
    \label{fig:tune}
\end{figure}

{\bf Towards multifunctional and tunable platforms.} After the rapid development of metaphotonics concepts based on passive dielectrics and semiconductors, recent trends in the field indicate strong shift towards materials with tunable optical properties and a wide range of functionalities~\cite{minovich2015functional,li2017nonlinear,makarov2017light,nemati2018tunable}. Figure~\ref{fig:Intro} schematically shows the emerging materials that enable the application of metaphotonics in such fields as adaptive flat optics, optical signal processing, optoelectronics, sensing, and data storage, where the outstanding optical properties of resonant nanostructures are highly desirable. Among the plethora of materials used for these applications, we focus on those most rapidly developing in the last years: phase change materials (PCMs), transition metal dichalcogenides (TMDCs), halide perovskites, upconversion nanoparticles (UCNPs), and so-called meta-shells. 

Despite their diversity, ultimately these materials are united by their inherent tunability that greatly contributes to the scope of possible applications, as external stimuli can be used to alter their characteristics and, thus, the functionality of the derivative devices. As detailed further, depending on the particular application, these materials can be ``primary'' (that is, when the metaphotonic structure itself is based on this material) or ``functionalizing'' (if it is added as a coating, substrate or a minor inclusion to a structure made of other material) while still sharing the common purpose of enabling the highly adaptive performance. In Figure~\ref{fig:tune}, we show a comparison of methods of tuning of these materials by various external stimuli and give an estimation of their applicability for each material. We also highlight the key mechanisms enabling the tunability, which is a cornerstone feature that guarantees high potential for the above mentioned real applications. Notably, the properties of these materials are quite complementary, which opens the possibilities on mutual beneficial integration, which we will discuss in the final part of the review.

In order to create functional metaphotonic nanostructures for a particular purpose, it is critical not only to select materials with the required properties, but also the most appropriate and compatible fabrication method. Figure~\ref{fig:property}a arranges the selected emerging materials depending on the prevalence of the certain type of the technique of their formation. Namely, PCMs are routinely used in the form of thin films, and their fabrication is based exclusively on  physical deposition methods such as magnetron sputtering \cite{wang2016optically}, while chemical synthesis is not applicable to creation of high-quality PCMs layers. TMDC bulk crystals and monolayers are also usually created \textit{via} physical methods. In particular, TMDC monolayers are usually fabricated \textit{via} manual exfoliation from bulk crystals grown using chemical transport \cite{ubaldini2013chloride} and are then relocated on various substrates \textit{via} stamp transfer technique. Recently, however, TMDC fabrication technologies had a turn towards more scalable methods such as chemical-vapor deposition\cite{shree2019high}. Furthermore, exfoliation of TMDCs can also be performed in liquids\cite{coleman2011two}, which enables further drop-casting or spin-coating deposition. In turn, chemical synthesis is much more popular for halide perovskites~\cite{dunlap2018synthetic}, where plenty of approaches were demonstrated to create nano- and micro-crystals in liquid~\cite{dey2021state} or on substrates~\cite{im20116, pushkarev2018few}, as well as to provide direct crystallization of liquid precursors on arbitrary surfaces and nanostructures~\cite{liang1998synthesis,era1997self, barrows2014efficient}. This makes halide perovskites equally suitable both as a primary material for direct metasurface fabrication~\cite{berestennikov2019active} and for functionalization of metastructures made of other materials~\cite{berestennikov2021enhanced}. Various chemical synthesis and coating methods are also utilized for functionalization of the primary metaphotonic designs. Namely, ligand/polymeric~\cite{staude2015shaping,chen2016functionalization,zyuzin2018photoluminescence,wu2020room} or metal-oxide-based (MO$_x$)~\cite{choi2015zinc, karvounis2020barium} layers can be added to optically resonant nanostructures made of high-index dielectric materials (e.g., Si, PCM, TMDC, \textit{etc.}).



\begin{figure}[t!]
    \centering
    \includegraphics[width=\linewidth]{Figs/Property.pdf}
    \caption{(a) Schematic diagram for the fabrication methods for materials and metaphotonic structures. (b) Schematic diagram for real refractive index dispersion of emerging materials for metaphotonics. Abbreviations: PCM - phase change materials, TMDC - transition metal dichalcogenides, UCNPs - upconversion nanoparticles. Low-index matrices include various metal-oxides and polymers used for meta-shell designs.}
    \label{fig:property}
\end{figure}


Generally, the materials with higher refractive index (similar to GeSbTe or TMDCs that have $n=3-5$ in the visible and near-IR spectral ranges) are more preferable as primary materials for metaphotonics because of the ability to strongly localize light. On the other hand, the low-index (n$<$2) materials are perfect for functionalization of metaphotonic nanostructures without critical modification of their optical properties, as they allow the preservation of high optical contrast. 
Figure~\ref{fig:property}b provides a comparison of refractive index values of the discussed functional and tunable emerging material platforms for metaphotonics, which is correlated with Figure~\ref{fig:property}a. Interestingly, the materials that are fabricated by physical methods tend to represent the ``high-index family'' and used as primary materials for metaphotonic structures, whereas chemically synthesized materials generally  possess lower refractive index, which is good for functionalization. From this perspective, TMDCs and halide perovskites represent a transitional case as they are being used in both instances.


\section{Phase change materials}

One of the most natural ways of achieving \textit{in-situ} tunable light interactions is temporal modulation of the refractive index. A transition from static to dynamic operation of metadevices that this feature implies would be a huge step towards practical implementations of nanostructured media, as it would allow addressing the issues of adaptive optics, dynamic beam steering for light detection and ranging, tunable holographic displays \textit{etc.} Such tuning is possible, for example, \textit{via} optical pumping or charge carrier implantation \cite{makarov2015tuning, huang2016gate} that leads to modification of the dielectric function due to the shift of the material plasma frequency or nonlinear optical effects. However, the modulation of refractive index in these cases is usually very minor. The development of tunable optical devices calls for material platforms supporting much higher refractive index modulation that would allow considerable spectral shift of the resonances or strong transmission/reflection modulation in metasurfaces and metamaterials. 

\subsection{Basic properties and tunability}


The recent decade witnessed the emergence and rapid evolution of the field of phase change nanophotonics. \cite{wuttig2017phase,hail2019optical,abdollahramezani2021dynamic} At the core of this direction are the so-called phase change materials, which exhibit structural phase transitions accompanied by substantial change of their physical properties, such as electric conductivity and refractive index (Fig.~\ref{fig:PCM-scheme}). Historically, at the forefront of these research efforts are two archetypal materials. The first is vanadium dioxide, which exhibits an insulator-metal transition (IMT) concurrent with a structural phase transition from monoclinic (M1, insulating) to tetragonal rutile (R, metallic) structure at T=340~K. The second is Ge$_2$Sb$_2$Te$_5$, or GST for short, and its various derivatives, which support amorphous-to-crystalline phase transition\cite{wuttig2017phase}. Both these material platforms offer fast (down to nanoseconds and beyond) and reversible switching of the phase, the discriminating factor being the volatility of this transition. While VO$_2$ returns back to its insulating state after the decrease of the temperature, GST can exist in both amorphous and crystalline states at normal conditions. It should be noted that the term ‘phase change materials’ originally corresponds to the non-volatile tunable materials rather than volatile ‘phase transition materials’ such as VO$_2$\cite{wuttig2017phase}. However, it has also been used in a broader sense to include both classes despite their drastically different mechanisms of optical contrast. 


\begin{figure}[t]
    \centering
    \includegraphics[width=0.9\linewidth]{Figs/PCM-Scheme-v1.png}
    \caption{
    %\textbf{by SINEV.} 
    Left: illustration of the structural phase change for VO$_2$ and chalcogenide PCMs. Center: characteristic spectral dependencies of the refractive index of VO$_2$ and Ge$_2$Sb$_2$Te$_5$ in two phase states. Red lines correspond to the states with higher optical losses (rutile for VO$_2$ and crystalline for GST), black lines - to more optically transparent monoclinic VO$_2$ and amorphous GST. Right: illustration of types of external stimuli used to induce the phase transition.}
    \label{fig:PCM-scheme}
\end{figure}


\textbf{Vanadium dioxide} represents a prime example of IMT in strongly correlated electron systems. During the first-order phase transition upon cooling, the material crystal symmetry is reduced from rutile to monoclinic (Fig.~\ref{fig:PCM-scheme}) \textit{via} dimerization of V atoms \cite{kubler2007coherent}. While the lattice symmetry in both states is well defined and confirmed experimentally, the exact mechanism of the transition is a subject of the ongoing debate with the main focus on whether it can be attributed to Peierls or Mott type \cite{liu2018recent}.  Furthermore, there is evidence that the IMT in VO$_2$ might support an intermediate monoclinic-like metallic phase and hints at a possible cooperative Mott-Peierls transition \cite{yao2010understanding,vidas2020does,grandi2020unraveling}. In any case, the demand for strong modulation of the refractive index ($\Delta$n$\approx$1 in the near-infrared spectral range) is met due to the phase transition between dielectric and metallic phases (Fig.~\ref{fig:PCM-scheme}). 

The chemical bonding in \textbf{chalcogenide PCMs} and its transformation during the crystalline-to-amorphous phase transition are also highly debatable due to their complexity arising from the strong electron delocalization \cite{kolobov2004understanding, lee2020chemical}. GST, as the most popular PCM, usually serves as the main subject for these studies. The complexity of this material from the chemical point of view manifests even for crystalline state, which has two modifications. The stable one represents a layered hexagonal crystal with the stacking sequence -Sb-Te-Ge-Te-Te-Ge-Te-Sb-Te-. However, based on X-ray diffraction (XRD) measurements, it was argued that crystallized GST in the form of thin films possesses the alternative metastable rocksalt-type structure with Te atoms occupying sites on one face-centred-cubic (f.c.c.) sublattice with Ge and Sb randomly forming the other f.c.c. sublattice with 20$\%$ vacant sites \cite{yamada2000structure}. The latter is preferable for the realization of reversible tuning. 

Extensive experimental studies of chalcogenide PCMs suggested that they do not fit into the common classification of materials by the bonding type (covalent, ionic and metallic). The characteristic parameters of the chemical bonds formed in these materials, such as effective coordination and effective Born charge, are intermediary with respect to classic examples of metallic and covalent types. Such bonds were therefore proposed to be assigned to a separate ``metavalent'' class \cite{zhu2018unique, wuttig2018incipient}.

The counter-intuitive coexistence of extremely fast crystallization rates and large contrast of optical properties ($\Delta$n$>$1.5 in the near-infrared spectral range for GST, Fig.~\ref{fig:PCM-scheme}) and resistivity (more than three orders of magnitude difference\cite{wuttig2007phase}) between the phases  also raises the question on the degree of affinity between crystalline and amorphous states of chalcogenide PCM. In the recent work, Lee et al. \cite{lee2020chemical} performed density functional theory calculations to identify very similar chemical-bonding nature in crystalline and amorphous states in the framework of the so-called hyper-bonding model.


\begin{figure}
    \centering
    \includegraphics[width=0.95\linewidth]{Figs/PCM-scheme-appl-v3.png}
    \caption{
    %\textbf{by SINEV.} 
    Classification of phase change material devices based on the type of integration, with the examples of applications enabled by integration of PCM with metaphotonics. (a) Plasmonic grating on Ge$_2$Sb$_2$Te$_5$ thin film for dynamic beam steering in the telecom spectral range. The plot shows the angle-resolved reflectance of the device for crystalline (red line) and amorphous (blue line) states of GST. Solid and dashed lines correspond to experiment and calculation, respectively. (b) Lens with a tunable focal distance based on Ge$_2$Sb$_2$Se$_4$Te$_1$ metasurface. One of the two superimposed USAF-1951 targets are imaged with the metalens depending on the phase state of GST. (c) Tunable Mie-resonant metasurface for mid-IR range based on hybrid Ge/GST disks. The spectra show gradual tuning of resonances in transmittance depending on the degree of GST crystallization by laser pulses. (d) Switchable linear polarizer based on doping-induced  VO$_2$ grating that changes its reflectance depending on the temperature. (a) Reprinted from Ref.~\citenum{galarreta2018nonvolatile} Copyright 2018 WILEY-VCH Verlag GmbH $\&$ Co. (b) Reprinted from Ref.~\citenum{shalaginov2021reconfigurable} Copyright 2021 Springer Nature. (c) Reprinted with permission from Ref.~\citenum{leitis2020all} Copyrignt 2001 John Wiley $\&$ Sons - Books. (d) Adapted from Ref.~\citenum{rensberg2016active} Copyright 2016 American Chemical Society.}
    \label{fig:PCM-scheme-appl}
\end{figure}

\subsection{Volatile and non-volatile functionalities}

In terms of integration with metaphotonic concepts, VO$_2$ and chalcogenide PCM compounds are being used both as a primary material for the resonators \cite{karvounis2016all,Tian2019active,kepic2021optically} and - in more rare cases - as a functional substrate that modifies the response of plasmonic or dielectric elements\cite{galarreta2018nonvolatile,tripathi2021tunable} as illustrated in Fig.~\ref{fig:PCM-scheme-appl}. The spectral range of large refractive index contrast between the phases of these materials extends from the visible to mid-IR, however, as we discuss further, many metaphotonic devices require careful management of losses which significantly narrows down the available options, especially in the case of VO$_2$ (Fig.~\ref{fig:PCM-scheme}). Furthermore, while similar functionalities based on the tuning of the refractive index can be achieved in metaphotonic devices based on both VO$_2$ and chalcogenide PCMs, in the following we consider them separately, discriminating the materials based on the volatility of the structural change depending on their applications.

{\bf Volatile metaphotonics with VO$_2$}. Being a volatile phase transition material, VO$_2$ lacks in certain aspects, in particular, energy efficiency (if the high temperature metallic state needs to be maintained), but offers undisputed advantages such as almost complete absence of volume change during the phase transition, superior chemical stability and extreme transition speed - below ps scale\cite{wegkamp2015ultrafast, jager2017tracking}. Due to its relatively low transition temperature, VO$_2$ is of particular interest for applications and architectures that are not compatible with intense pulsed lasers or high local temperatures.

The phase transition of VO$_2$ can be triggered thermally, optically and also \textit{via} doping. In the latter case, it is possible to reduce the already low phase transition temperature even further by $30^\circ$C. Rensberg with co-authors\cite{rensberg2016active} (Fig.~\ref{fig:PCM-scheme-appl}d) modified the VO$_2$ film by ion-beam irradiation, and the temperature lowering was controlled by the feature size of a poly(methyl methacrylate) mask (Fig.~\ref{fig:PCM-scheme-appl}d). Electrical switching driven by the heating of a metallic electrode was demonstrated in the work by Hashemi \textit{et al.} \cite{hashemi2016electronically}
Despite considerable losses in both insulating and metallic states ($k>0.5$) in the spectral region of maximum contrast of the real part of the refractive index (Fig.~\ref{fig:PCM-scheme}), VO$_2$ was successfully used for tunable metasurfaces either as a primary material \cite{kepic2021optically} or as a functional addition to silicon resonators.\cite{howes2020optical,tripathi2021tunable}. Tripathi and co-authors\cite{tripathi2021tunable} demonstrated tunable spectral filtering in a Si-based resonant metasurface functionalized by a VO$_2$ substrate. Howes \textit{et al.} integrated a thin VO$_2$ layer on top of silicon resonators to achieve tunable Huygens regime for a hybrid metasurface\cite{howes2020optical}. A transition from dielectric to plasmonic Mie resonances in VO$_2$ disks driven by CW laser irradiation was demonstrated by Kepi\v{c} and co-authors in Ref.~\citenum{kepic2021optically}.

{\bf Non-volatile metaphotonics with chalcogenide PCMs}. Chalcogenide PCMs were known for decades in the fields of optical and electronic data storage. The prefix ``chalcogenide'' refers to the fact that these compounds invariably contain the elements from group VIa of the periodic table – Te, Se, or S. The amorphous phase of these materials is usually characterized by high electrical resistivity (10$^7\Omega_{sq}$ for GST) and low optical losses below the band gap energy. Their crystallization leads to a decrease in the bandgap energy, concurrent with the increase of optical losses and a considerable drop in the resistivity (three orders of magnitude for GST\cite{friedrich2000structural}). Such a prominent transformation made PCMs (GST, AIST) the cornerstone materials for memory applications \cite{wuttig2007phase,driscoll2009memory}. Furthermore, the switching of chalcogenide PCM state can be performed with appropriate thermal, electric or optical stimuli, which facilitated their integration  into both electronic and optical commercial technologies.

High contrast of the optical properties between the phases called for the application of PCM in nanophotonic devices. First attempts to create phase change metasurfaces involved direct imprinting of phase state mask into thin ($<$100~nm) PCM films using laser pulses, which allowed to create devices such as Fresnel lenses \cite{wang2016optically}. Later, high sensitivity of surface plasmon resonance to the local dielectric permittivity inspired the integration of PCM films with plasmonic resonators \cite{galarreta2018nonvolatile} to accomplish beam steering devices in the near-infrared spectral range (Fig.~\ref{fig:PCM-scheme-appl}a).

More recently, it was realized that chalcogenide PCMs fit perfectly into the concept of all-dielectric nanophotonics. In particular, optical properties of GST in its amorphous state closely match those of silicon, which allowed to directly emulate various designs developed earlier during the rise of all-dielectric nanophotonics\cite{kuznetsov2016optically}. In one of the first works in this direction, Tian et al. demonstrated tunable anapole states in mid-IR spectral region for a metasurface made of GST disks \cite{Tian2019active}. Shortly after, hybrid silicon-GST and germanium-GST metasurfaces were used to achieve tunable spectral filtering performance  in near- and mid-IR \cite{leitis2020all,galarreta2020reconfigurable} (Fig.~\ref{fig:PCM-scheme-appl}c). Fabrication and reversible tuning of Mie-resonant GST nanoparticles using laser ablation was demonstrated by Rybin \textit{et al.} \cite{rybin2021optically}. However, due to the diversity of chalcogenide PCMs, their functionality is not limited to dielectric concepts. In particular, the recently emerged phase change material In$_3$SbTe$_2$ exhibits a transition between metallic and dielectric behaviour in the mid-infrared spectral range, which allows for direct laser imprinting of resonant metallic elements in thin films without the need for multi-stage lithography process\cite{Hessler2021}.

It should be noted that there are several viable rivals to phase change material platform in the field of tunable metaphotonic devices. One of them are liquid crystals (LCs), which provide a contrast of the refractive index when subjected to elevated temperatures\cite{komar2018dynamic} or external electric fields\cite{komar2017electrically} that lead to re-orientation of the highly anisotropic LC molecules. While holding certain advantages over PCMs (in particular, the strong background of the well-developed liquid crystal display technologies\cite{li2019phase}), LCs have drawbacks in the spectral range of operation (limited to visible and near-IR wavelengths), maximum refractive index contrast (defined by LC birefringence $n_o-n_e$=0.2) and switching speed (up to milliseconds). Another alternative is chemical and electrochemical switching of metasurfaces. In particular, controlled hydrogenation and dehydrogenation of Mg nanoparticles was used for dynamic plasmonic color displays\cite{Duan2017dynamic}. Very recently, electrochemical reactions were utilized for color tuning in TiO$_2$-based metasurfaces\cite{eaves2022dynamic} and for beam steering with metallic polymer nanoantennas in the near-infrared\cite{karst2021electrically}. However, not unlike LCs, these approaches are quite limited in the modulation speed (from 10s of milliseconds\cite{karst2021electrically} up to minutes\cite{Duan2017dynamic,eaves2022dynamic}).


\begin{figure}
    \centering
    \includegraphics[width=0.75\linewidth]{Figs/PCM-outlook-v2.png}
    \caption{
    %\textbf{by SINEV.} 
    Recent development trends in the field of phase change metaphotonics. (a) Emerging large band gap PCM. Left: Refractive index of Ge$_2$Sb$_2$Se$_x$Te$_{5-x}$ for different x and the demonstration of a planar SiN ring waveguide with Ge$_2$Sb$_2$Se$_4$Te$_1$ patch working as a modulator in the telecom range. Right: dispersion of the refractive index for Sb$_2$S$_3$ and Sb$_2$Se$_3$. Demonstrations of tunable structural color generation in the visible range in an Sb$_2$S$_3$ metasurface and an Sb$_2$S$_3$ thin film.  (b) Reversible multilevel tuning of a hybrid Si/GST metasurface: experimentally measured (center) and numerically calculated (right) maps of the device reflectance depending on the phase state of GST layer. (c) Electrically driven switching of PCM metasurface. Left: scheme of the device. Right: simulated and measured reflectance spectra of the metasurface after electrically driven switching. (a) Reprinted with permissions from Ref.~\citenum{zhang2019broadband} Copyright 2019 Springer Nature, Ref.~\citenum{delaney2020new} Copyright 2020 WILEY-VCH Verlag GmbH $\&$ Co, Ref.~\citenum{lu2021reversible} Copyright 2021 American Chemical Society and from Ref.~\citenum{liu2020rewritable}. \textcopyright  The Authors, some rights reserved; exclusive licensee AAAS. Distributed under a CC BY-NC 4.0 license http://creativecommons.org/licenses/by-nc/4.0/.  Reprinted with permission from AAAS. (b) Reprinted from Ref.~\citenum{galarreta2020reconfigurable} Copyright 2020 The Optical Society. (c) Reprinted with permission from Ref.~\citenum{zhang2021electrically} Copyright 2021 Springer Nature BV. }
    \label{fig:PCM-outlook}
\end{figure}

\subsection{Emerging applications}

Despite the good optical contrast of GST (Fig.~\ref{fig:PCM-scheme}), its initially small band gap (0.5 eV in its crystalline state) and strong losses in the crystalline state ($k\approx2$ at 1000~nm) limit its applicability to mid- and, in particular cases, near-infrared spectral ranges. As the field of all-dielectric nanophotonics always gravitated towards the visible spectral range, the research focus in the last few years shifted towards larger bandgap PCMs (Fig.~\ref{fig:PCM-outlook}a). It was found that gradual substitution of Te with Se in GST leads to the increase of the bandgap without the loss of the fascinating phase change behavior. New compounds, in particular, Ge$_2$Sb$_2$Se$_4$Te$_1$, support low-loss performance of phase change devices in the mid-IR range in both states (Fig.~\ref{fig:PCM-outlook}a), which enabled highly efficient modulators for integrated circuits \cite{zhang2019broadband}  and tunable metalenses\cite{shalaginov2021reconfigurable} (Fig.~\ref{fig:PCM-scheme-appl}b).

Most recently, binary antimony compounds – Sb$_2$Se$_3$ and Sb$_2$S$_3$ – emerged as promising candidates for bringing the phase change functionality all the way to the tantalizing visible spectral range \cite{dong2019wide,lu2021reversible}. With their considerably larger bandgap as compared to GST (1.2~eV for Sb$_2$Se$_3$ and 2.05 eV for Sb$_2$S$_3$), these materials exhibit negligible losses in both phase states above 1~$\mu$m wavelength. Some of the very recent works demonstrate vivid tunable structural color generation with metasurfaces made of Sb$_2$Se$_3$ and Sb$_2$S$_3$\cite{lu2021reversible,hemmatyar2021enhanced} (Fig.~\ref{fig:PCM-outlook}a).


For the full-scale implementation of metadevices enabled by chalcogenide PCMs there are several challenges to be addressed. Arguably, the main issue is the reversibility of the phase switching, which is quite rarely demonstrated experimentally\cite{galarreta2020reconfigurable,rybin2021optically} (Fig.~\ref{fig:PCM-outlook}b).  Since the amorphization requires rapid quenching from the molten state, the reversible switching is facilitated if the design of phase change metasurfaces contains efficient heat sinks, as PCM themselves usually have quite low thermal conductivity ($<$0.5W/m/K for GST). In particular, this precludes the use of large volumes of PCM if a reversible functionality is desired. Binary antimony compounds also suffer from segregation (see Ref.~\cite{delaney2020new} and references therein) and significant changes of the material volume during the switching process, which mitigates the effective spectral shift of the optical resonances. Furthermore, lots of applications, such as \textit{e.g.} dynamic beam steering, require continuous tuning of the optical response rather than discrete on/off states. Recently, multilevel tuning of optical resonances was demonstrated in a hybrid silicon-GST metasurface operating in the near-IR spectral range\cite{galarreta2020reconfigurable} (Fig.~\ref{fig:PCM-outlook}b).

Finally, practical applications require the compact integration of phase change metasurfaces with units that provide the switching of the phase state. From this standpoint, electrically driven switching enabled by contact grids provides the necessarily small device fingerprint as well as the possibility of single-pixel addressing. Latest research has demonstrated reversible tuning of PCM metasurfaces \textit{via} electrically driven heating \cite{wang2021electrical, zhang2021electrically,abdollahramezani2022electrically} (Fig.~\ref{fig:PCM-outlook}c).

\section{Transition metal dichalcogenides}

\subsection{Basic properties and tunability}

Since the discovery of the extraordinary properties of graphene, enormous research attention funneled into the field of 2D atomically thin crystals~\cite{dai2020artificial}. Among the numerous kinds of such systems we will focus on transition metal dichalcogenides (TMDCs), in particular, selenides and sulphides of tungsten and molybdenum (sharing a common stoichiometry MX$_2$, Fig.~\ref{fig:TMDC1material}a,b), which arguably became the most popular 2D material platform for optical applications. The reason for this particular interest is that these materials are layered semiconductors that have a direct bandgap in the monolayer limit (Fig.~\ref{fig:TMDC1material}c)~\cite{splendiani2010emerging, mak2016photonics} that enables photoluminescence with potentially near-unity quantum yield~\cite{amani2015near, lien2019electrical}. Many fascinating properties of TMDCs stem from the fact that they support excitons (bound electron-hole pairs) with huge binding energy (typical values of the order of 0.5~eV) stipulated by their two-dimensional nature and suppressed Coulomb screening~\cite{wang2018colloquium} (see Fig.~\ref{fig:TMDC1material}e for low-temperature excitonic PL spectra of standard TMDCs). Furthermore, TMDCs manifest strong spin-orbit coupling (with maximum spin splittings measuring hundreds of meV~\cite{zhu2011giant}) that leads to spin-valley locking~\cite{xu2014spin}: optical pumping with circularly polarized light enables selective exciton population of one of the two valleys at the edges of the electronic Brillouin zone depending on the light helicity (Fig.~\ref{fig:TMDC1material}d). The possibility of further optical readout of the stored valley information is defined by the valley polarization contrast of photoluminescence that can be enhanced \textit{e.g.} in the presence of chiral colloidal particles.\cite{kim2020single}  Excitonic nature of TMDCs defines their excellent tunability that can be achieved by multiple means, including gate voltage\cite{ross2013electrical,van2020exciton} (Fig.~\ref{fig:TMDC1material}f) and magnetic field\cite{li2014valley} (Fig.~\ref{fig:TMDC1material}g), optical pumping\cite{barachati2018interacting,kravtsov2020nonlinear} and control of dielectric environment\cite{raja2017coulomb}. Furthermore, despite relative chemical robustness of TMDCs, multiple ways of their chemical functionalization were proposed\cite{chen2016functionalization}.

Such properties (summarized in Fig.~\ref{fig:TMDC1material}) promise exciting physics of TMDCs arising in both weak and strong light-matter coupling regimes (as detailed further) - from light-emitting and valleytronic applications to single-photon level nonlinearities - and inspired strong integration with the rapidly developing field of metaphotonics\cite{krasnok2018nanophotonics}. The main rival of TMDCs in these fields are quantum well structures based on classical semiconductors (GaAs, GaN etc.). However, low exciton binding energies in these systems (10s of meV) confines their operation to cryogenic temperatures, which significantly hinders their practical applications.

\begin{figure}
    \centering
    \includegraphics[width=0.9\linewidth]{Figs/TMDC_1-material.png}
    \caption{
    %(SINEV) 
    (a) Photo of a WS$_2$ bulk crystal. (b) Schematic view of the crystalline structure of a TMDC monolayer. (c) Calculated band structures of bulk, quadrilayer, bilayer and monolayer of MoS$_2$. (d) Schematics of the band structure and optical selection rules of excitons in monolayer TMDC. (e) Photoluminescence spectra of TMDC monolayers at 4K when deposited directly onto SiO$_2$ (top) and when capped with hBN (bottom). (f) Photoluminescence spectra of a MoSe$_2$ monolayer as a function of the back-gate voltage. At zero doping level, neutral excitons (X$^0$, X$^I$) dominate the spectrum. With large electron (hole) doping, negatively (positively) charged excitons (trions) take over. (g) Representation of the circularly polarized PL spectra of a monolayer MoSe$_2$ as a function of the applied magnetic field. (c) Reprinted with permission from Ref.~\citenum{splendiani2010emerging} Copyright 2010 American Chemical Society. (d) Reprinted with permission from Ref.~\citenum{sun2019separation} Copyright 2019 Springer Nature BV. (e) Reprinted with permission from Ref.~\citenum{cadiz2017excitonic} Copyright 2017 American Physical Society. (f) Reprinted with permission from Ref.~\citenum{ross2013electrical} Copyright 2013 Springer Nature. (g) Reprinted with permission from Ref.~\citenum{li2014valley} Copyright 2014 American Physical Society.}
    \label{fig:TMDC1material}
\end{figure}

\subsection{Light-matter coupling with TMDCs}

Layered 2D crystals, including TMDCs, are particularly attuned to metasurfaces due to their shared planar nature. This enables many technological approaches \cite{mupparapu2020integration}, including integration with pre-fabricated photonic designs and direct patterning (Fig.~\ref{fig:TMDC2scheme}). Excitons in TMDCs (``matter'') can efficiently couple with the optical modes (``light'') of planar structures. This interaction is conventionally treated as a pair of coupled oscillators and classified into weak and strong coupling regimes. The transition between the regimes is defined by the balance between the intrinsic spectral linewidths of the exciton ($\gamma$) and cavity mode ($\kappa$) resonances and the coupling constant $g$ which is determined by the system geometry and the exciton oscillator strength (which is exceptionally high for TMDCs as mentioned before). Strong coupling criterion is defined as
\begin{equation}
 g >  |\gamma-\kappa|/2,
\end{equation}
while the opposite case $g \leq  |\gamma-\kappa|/2$ is referred to as weak coupling.  In the \textit{weak} light-matter coupling regime, the excellent light-emitting properties of TMDCs can be further enhanced using metasurfaces as they allow control of both the local density of optical states and the directionality of the radiation. Metasurfaces based on non-metallic materials, while limited in the maximum field amplification as compared to their plasmonic counterparts, facilitate independent engineering of both absorption and emission of TMDCs and are practically free from PL quenching effects. Using this approach, Zhang et al. demonstrated a 1300-fold increase of the photoluminescence intensity from a MoS$_2$ monolayer on a suspended silicon nitride photonic crystal designed to possess spatially extended Fano resonances \cite{zhang2017unidirectional}. Metasurface design also allows to manipulate the temporal dynamics of the emission through Purcell effect as shown by Chen et al. for a WSe$_2$ monolayer coupled to a silicon metasurface \cite{chen2017enhanced}. Bucher et al. observed strong (up to two orders of magnitude) enhancement, as well as spectral and directional reshaping of emission of MoS$_2$ integrated with a silicon metasurface \cite{bucher2019tailoring}.

\begin{figure}
    \centering
    \includegraphics[width=0.95\linewidth]{Figs/TMDC-Scheme_v2.png}
    \caption{
    %(SINEV) 
    Current technology and applications in the field of TMDCs integrated with metaphotonic structures. (a) Integration of exfoliated TMDC monolayers with resonant photonic structures. Adapted with permission from Ref.~\citenum{bucher2019tailoring} Copyright 2019 American Chemical Society. (b) Direct patterning of bulk and monolayer TMDC crystals using self-terminating wet etching process. Adapted with permission from Ref.~\citenum{munkhbat2020transition} Copyright 2020 Springer Nature. (c) Orientation-controlled stacking of multiple monolayers of 2D crystals that enables the engineering of the landscape of the electronic potential. The figure shows a dark-field scanning transmission electron microscopy (STEM) image of a MoS$_2$ homobilayer. Adapted with permission from Ref.~\citenum{weston2020atomic} Copyright 2020 Springer Nature BV. (d-f) Illustrations of various applications of TMDCs in the regimes of weak and strong coupling with optical modes of metaphotonic structures inspired by Refs.~\citenum{bucher2019tailoring, sun2019separation, liu2020generation}: (d) Metasurface-assisted enhancement of light generation with TMDCs; (e) Spatial separation of valley-polarized excitons pumped with circularly polarized light of different helicity ($\sigma_+/\sigma_-$) (f) One-way guiding of exciton polaritons \textit{via} topologically protected edge modes.}
    \label{fig:TMDC2scheme}
\end{figure}

Metasurface concepts can also be applied for the optimization of second harmonic generation, which is resonantly enhanced at the exciton frequency of TMDCs~\cite{malard2013observation}. In this case, the most critical parameter is the enhancement of the optical field at the pump wavelength, which can be provided by high-Q resonances of the metasurface. L{\"o}chner et al. utilized an asymmetric unit cell silicon metasurface supporting quasi-bound states in the continuum to improve the SHG signal from MoS$_2$ \cite{loechner2020hybrid}.

\textit{Strong} light-matter coupling regimes with TMDCs is associated with the formation of hybrid half-light half-matter quaiparticles - exciton polaritons. First realizations of this regime in TMDCs were achieved by enclosing the monolayers in vertical Fabry-P{\'e}rot cavities \cite{liu2015strong}. However, such constructions severely lack the means for fine tuning of the optical modes and are technologically quite difficult to integrate. Metasurfaces, as planar photonic cavities, are naturally compatible with layered semiconductors and offer unprecedented flexibility for control of the photonic part of the exciton polaritons. 

Within the last few years, the research on metasurface-coupled TMDCs reproduced several milestone phenomena that were earlier achieved in the more mature field of vertical cavity TMDC polaritonics. The key starting point - strong light-matter coupling and formation of exciton polaritons in planar photonic cavities -  was accomplished in the work by Zhang et al. \cite{zhang2018photonic}, where a WSe$_2$ monolayer integrated with a photonic crystal slab exhibited strong exciton-photon coupling up to room temperature. Another crucial aspect of TMDC polaritons, nonlinearity, was shown in a metasurface supporting bound states in the continuum coupled to a MoSe$_2$ monolayer \cite{kravtsov2020nonlinear} with record high values. Finally, lasing action was achieved in various TMDC-based systems such as photonic crystal cavity coupled to a monolayer of WSe$_2$ \cite{wu2015monolayer},  microflakes of a less common 2D material InSe\cite{li2022room} and in a silicon nitride metasurface integrated with an aligned MoSe$_2$/WSe$_2$ heterostructure using interlayer excitons as gain medium \cite{paik2019interlayer}. 

While providing strong optical field localization required for polaritonics, metasurfaces also grant ultimate control over both dispersion and polarization state of the optical modes that allows reaching far beyond the capabilities of vertical cavities. In particular, spin-momentum locking that is achieved with specially engineered planar photonic designs opens the route towards valleytronics\cite{schaibley2016valleytronics}.  The prerequisite for building actual valleytronic devices - the spatial separation of valley-polarized excitons - was demonstrated by Sun et al. \cite{sun2019separation} with the asymmetric groove plasmonic metasurface. Another way to realize valley-momentum locking is offered by helical topological edge states. Works by Liu et al. \cite{liu2020generation} and Li et al. \cite{li2021experimental} demonstrated different regimes of coupling between TMDC excitons and topological photonic modes (Fig.~\ref{fig:TMDC2scheme}f) for WS$_2$, WSe$2$ and MoSe$_2$ covering the spectral range from 600 to 800~nm. Large topological gap enabled by the photonic design in these works allowed the realization of a topological polaritonic phase without the need for time reversal symmetry breaking with the magnetic field that was shown earlier for quantum well structures~\cite{klembt2018exciton}. While photonic and even polaritonic topological edge states can be achieved on different material platforms, including light-emitting ones such as perovskites (see Section~\ref{sec:pero}), valley-resolved operation available for TMDCs significantly enriches the scope of applications. Furthermore, long valley depolarization times of TMDCs (as long as 6~ps reported for WSe$_2$\cite{zhu2014exciton}) allow to readout the valley information coupled to propagating edge modes\cite{liu2020generation,li2021experimental}.

\subsection{Emerging technologies}

Bulk TMDC crystals, while lacking in the fascinating excitonic properties of the monolayers due to bandgap renormalization and increased Coulomb screening, also found their applications in nanophotonics. Despite extensive research efforts in the field, it was not until recently that the huge in-plane refractive index of TMDCs (over 4 in the visible and near-IR spectral ranges \cite{beal1979kramers}) was fully appreciated. The first applications of huge effective optical path length of TMDCs involved the creation of ultrathin lenses and gratings \cite{yang2016atomically, liu2018ultrathin}. The same realization inspired Verre \textit{et al.} to pattern a bulk WS$_2$ crystal to create nanodisks which showed high-Q Mie resonances  in the visible spectral range \cite{verre2019transition}. The size-dispersive character of the resonances also allowed the observation of the strong coupling between the WS$_2$ exciton and the anapole mode of the disk.  

Furthermore, due to their layered structure which combines strong intralayer covalent bonding with weak interlayer van-der-Waals interactions, TMDCs demonstrate extreme anisotropy of their refractive index. In a recent paper, Ermolaev \textit{et al.} \cite{Ermolaev2021} performed a comprehensive experimental study supported by the first principle calculations that revealed a huge birefringence of 1.5 in the infrared and 3 in the visible spectral range for MoS$_2$. Even higher refractive index (n$>$5) and its anisotropy is expected for less conventional TMDC, ReSe$_2$, as predicted by Shubnic et al. using density functional theory calculations\cite{shubnic2020high}. 

Recently developed methods of nanostructuring of bulk TMDCs establish them as an extremely promising material for metaphotonics. The combination of high refractive index and strong nonlinear response driven by the excitonic nature of TMDCs already inspired several realizations of metasurfaces and nanoantennas for harmonic generation\cite{nauman2021tunable}. However, potential applications go way beyond the field of nonlinear optics, as high refractive index by itself empowers metasurfaces for structural color generation, waveguides, photonic crystal slabs \textit{etc.} \cite{munkhbat2022nanostructured} Unique lattice structure of TMDCs also facilitates advanced fabrication routines with fine control over the lateral etching process. Munkhbat \textit{et al.} found that anisotropic wet etching using hydrogen peroxide applied to TMDs induces a self-terminating process resulting in the formation of hexagonal structures of predefined order and complexity.\cite{munkhbat2020transition}

Further integration of TMDCs with meta-photonic devices requires scalable fabrication technologies. Despite enormous progress in large-scale TMDC fabrication methods, such as chemical vapor deposition (CVD) \cite{shree2019high} and molecular beam epitaxy (MBE) \cite{fu2017molecular}, manual exfoliation for now remains the most common method for production of high quality TMDC monolayers. Up until recently, the main limiting factor for CVD and MBE was the misorientation of crystallographic domains during the growth which resulted in high defect density. In their recent work, Li \textit{et al.} overcame this issue by growing a wafer-scale single MoS$_2$ crystal on a sapphire substrate with specifically designed miscut angle \cite{li2021epitaxial}. Still, the subsequent transfer of TMDCs from growth substrates to photonic structures is difficult due to the strong bonding between the materials, which precludes the use of dry transfer techniques suitable for exfoliated crystals \cite{castellanos2014deterministic}. Ref.~\citenum{gurarslan2014surface} suggested an alternative wet transfer technique for CVD and  MBE-grown TMDCs that leverages the penetration of water between hydrophobic TMDC layers and hydrophilic growth substrates.

An alternative approach for construction of TMDC-based metasystems involves meticulous arrangement of 2D crystals into compound multilayer stacks called van-der-Waals heterostructures \cite{geim2013van}. Fine control over the relative orientation of the successive layers allows the creation of unique electronic potentials such as Moir{\'e} superlattices~\cite{tran2019evidence, seyler2019signatures} which contributes to a broader research direction termed ``twistronics''\cite{carr2017twistronics}. The diversity of nanoelectronic properties that are achieved for heterostructures of different stacking are revealed using scanning transmission electron microscopy studies~\cite{weston2020atomic}. These artificial excitonic crystals open yet another pathway for control of light-matter interactions, creation of arrays of identical quantum emitters, and beyond~\cite{tran2020moire}. 

\section{Halide perovskites}
\label{sec:pero}

Halide perovskites is a rapidly developing class of semiconductors, which can be synthesised both by wet-chemistry approaches and by various physical deposition methods. They have been intensively studied due to their outstanding electronic and optical properties, which are extremely prospective for solar photovoltaics~\cite{stranks2015metal,jena2019halide,li2020perovskite}, light-emitting devices~\cite{sutherland2016perovskite, liu2021metal, gets2021reconfigurable}, photodetectors~\cite{dou2014solution,ahmadi2017review,marunchenko2021single}, X-ray detectors~\cite{wei2016sensitive}, as well as for data storage and memristors~\cite{zhumekenov2021stimuli,liu2022nanostructured}.

Similarly to TMDCs, most of halide perovskite compositions support excitons at room temperature with binding energies in the range of 10-300~meV~\cite{d2014excitons,mauck2019excitons,baranowski2020excitons}, which cause their high quantum yield of photoluminescence, electronic density of states, and, thus, high optical gain, making them very promising materials for lasing applications~\cite{zhu2015lead, sutherland2016perovskite,zhang2017advances,qin2020stable,lei2021metal}. Moreover, such a strong excitonic state provides an additional resonant-like increase of refractive index, which  opens up additional opportunities for perovskite optical modulators and sensors, as well as allows them to support pronounced Mie resonances in the visible and near-infrared ranges~\cite{tiguntseva2018light,tiguntseva2020room}. 
%overall about possible transformation in the perovskites

A generic perovskite has a crystal structure and composition ABX$_3$, where `A' and `B' are cations and `X' is an anion (Figure \ref{fig:pero1}a). Cation `A' is often represents Cs$^+$ or CH$_3$NH$_3^+$, and is bigger than cation `B'. The latter largely defines the scope of applications of the perovskite: Pb$^+$ as cation `B' provides best device performance, Sn$^+$ is used for the reduction of  toxicity~\cite{xiao2019lead}, Ge$^+$ tends to exhibit strong second-order nonlinear response~\cite{stoumpos2015hybrid}. The most stable chemical composition is cesium lead halide perovskite (CsPbX$_3$). Anion `X' in this case is I$^-$, Br$^-$, or Cl$^-$. Perovskite lattices can have different types of crystal phases (Figure~\ref{fig:pero1}b): cubic, tetragonal and orthorhombic. Similarly to the most of semiconductors, optical properties of halide perovskites can be changed by the dimensionality or level of confinement of the electron wave-function: 0D - nanocrystals, 1D - nanowires, and 2D - monolayers (Figure~\ref{fig:pero1}c). Moreover, it is possible to assemble a superlattice from perovskite nanocrystals - a bulk 3D structure with unique quantum properties~\cite{raino2018superfluorescence}.
Due to ionic bonding of atoms in perovskite crystal, mixed-halide perovskites support spatial composition separation (so-called segregation~\cite{hoke2015reversible}) that is manifested by the application of an external electric field or irradiation by relatively intense light, which is an additional opportunity for tuning (Figure~\ref{fig:pero1}d).
Another approach to change the material optical properties is to fabricate structures of special geometry that support optical resonances.

\subsection{Basic properties and tunability}

\textbf{Composition change.}
The most straightforward way of transformation of perovskites is tuning of the band gap by the change of chemical composition (Fig.~\ref{fig:pero1}a). According to the \textit{ab initio} calculations, the electronic states related to cation A (CH$_3$NH$_3^+$, Cs$^+$, etc.) are far from the band edges~\cite{yin2015halide} and, thus, the change of the band gap by replacement of this cation is slight. The conduction band is formed by the Pb \textit{s}-orbitals, whereas the valence band is formed by Pb \textit{s}-orbital and \textit{p}-orbital of X (I, Br, Cl) which has an antibonding character. Thus, the dual nature of halide perovskite electronic structure is unique: the conventional semiconductors have a conduction band formed by \textit{s}-state and the valence band formed by \textit{p}-state. By replacement of the anion X, the emission wavelength can be tuned in the whole visible range\cite{sutherland2016perovskite}. The shortest emission wavelength of 425~nm was obtained in CsPbCl$_3$~\cite{protesescu2015nanocrystals}. By the exchange of Cl$^-$ with Br$^-$, the luminescence wavelength of this material can be tuned in the range from 425~nm to 530~nm. Further, by the replacement of Br$^-$ with I$^-$, the emission wavelength is shifted up to 700~nm. The subsequent shift up to a value of 770~nm can be achieved by the use of the CH$_3$NH$_3$ instead of Cs\cite{zhao2017high}. In order to create narrow-band halide perovskites operating in the near-IR range similarly to silicon ($\sim$1~eV), Sn-based organic-inorganic compositions were synthesized and optimized~\cite{chang2020tunable}.

%In order to avoid the use of Pb, the perovskite with Sn$^-$ as B-cation can be synthesized~\cite{han2019lead}. For example, Cs$_2$SnX$_6$ (X = Br, I) have a good PL yield and well-defined crystal structure. In this case, Cs$_2$SnBr$_6$ possessed the maximum photoluminescence emission at 673~nm, whereas maximum photoluminescence emission of Cs$_2$SnI$_6$ locates at 870~nm. For the purpose of synthesis of a lead-free stable compound, the new double perovskite was developed~\cite{luo2018efficient}. By alloying sodium cations into Cs$_2$AgInCl$_6$ the dark transition was broken which leads to an increase in photoluminescence efficiency by three orders of magnitude compared to pure Cs$_2$AgInCl$_6$. Moreover, the broad emission bandwidth allows to consider these structures as a white-light source. 

\begin{figure}[h!]
    \centering
    \includegraphics[scale=0.7]{Figs/fig1_perovskite.pdf}
    \caption{
    %\textbf{by TONKAEV.} 
    Transformation mechanisms for halide perovskites. (a) Transformation of the band gap \textit{via} change of the anions. (b) Change of the perovskite lattice structure \textit{via} temperature. (c) Change of the dimensionality from 0D nanocrystals to bulk materials and supperlattices assembled from ordered nanocrystals. (d) The concept of spatial composition control by the application of external electric field and light due to segregation. %Assembling of structures with unique optical properties from perovskite resonant nanoparticle}
    }
    \label{fig:pero1}
\end{figure}

%
\textbf{Phase change.}
The electronic properties, optical response, and luminescence of the perovskites is strongly connected to their crystalline structure. The research on the aspects of phase transition in the halide perovskite has been conducted throughout the last decades\cite{hirotsu1971experimental, hirotsu1974structural, hirotsu1978elastic, fujii1974neutron,tovborg1969nqr}. 
Due to the peculiarities of the perovskite crystal structure, only a few crystallographic structures can exist: orthorhombic, tetragonal, and cubic. 
The possible crystal lattices are presented in Fig.~\ref{fig:pero1}b. The cation B (yellow) is located in the centre of a parallelepiped formed by cation A (pink) and is surrounded by an octahedron of X anions (purple). The highest symmetry is observed when the length of the connections between cations A have the same length (a = b = c), which corresponds to a cubic phase and is observed at the elevated temperatures. With the breaking of the symmetry and transition to the case when only two of the unit cell lengths are equivalent (a = b $\neq$ c), the perovskite phase changes to tetragonal, which usually manifests near room temperature. In the case when all three unit cell lengths are different (a $\neq$ b $\neq$ c), perovskites have orthorombic symmetry, which is usually observed at low temperatures. 
All these phases occur in perovskites with arbitrary chemical composition, including mixed-halide perovskites. 

There are two main phase transitions in one of the most popular organic-inorganic perovskite compositions CH$_3$NH$_3$PbI$_3$~\cite{whitfield2016structures}. The first one is from cubic to tetragonal, which appears at the temperature of T=330~K and leads to a weak shift of the photoluminescence peak. Whereas the transition from tetragonal to orthorhombic lattice at T=160~K causes a sharp band gap blue-shift of 0.2~eV. Remarkably, the nanoimprinted CH$_3$NH$_3$PbI$_3$ grating exhibits even stronger sensitivity of the photoluminescence to the temperature change due to the coupling with the optical mode after the phase transition~\cite{tiguntseva2019enhanced}. In bulk CH$_3$NH$_3$PbBr$_3$, the same transitions are observed, however, there is a slight difference in the cubic to tetragonal transition temperature (T=130-150~K) and a more significant difference for tetragonal to orthorhombic transition (T=230-250~K)\cite{liu2018temperature}. However, these transitions are weakly manifested in photoluminescence.

Inorganic CsPbX$_3$ (X = Cl, Br, I) perovskites also possess orthorhombic phase at low temperature and cubic phase at high temperature. However, they also have several types of tetragonal phases\cite{hirotsu1971experimental, hirotsu1978elastic, tovborg1969nqr}.
For example, CsPbCl$_3$ changes its lattice from tetragonal to cubic at T=320~K, and transitions between different tetragonal phases with various symmetries at T~=~315~K and T~=~310~K~\cite{hiraoka2005observations}. Nevertheless, a significant change of photoluminescence was observed only in CsPbCl$_3$ upon its transition from tetragonal to orthorhombic phase (T~=~200~K) \citep{yi2020correlation}. While the photoluminescence peaks of CsPbBr$_3$ and CsPbI$_3$ demonstrate a gradual blue-shift with increasing temperature, the emission peak of CsPbCl$_3$ changes its behavior at T~=~200 K and a red-shift appears at higher temperature.

% 
\textbf{Dimensionality change.}
% from bulk to quasi 2D
Generally, in order to change the optical properties of semiconductors, one can reduce the dimensionality from bulk to 2D, 1D, and 0D cases thus confining the electron wavefunction. 
Quasi-2D perovskites are rapidly developing category of perovskites which represent self-assembled multi-quantum-well structures. These perovskites have been applied in photonics, light-emitting diodes (LEDs) and solar cells due to their outstanding optoelectronic  properties~\cite{byun2016efficient,cheng2020multiple,zhang2021high}, and many similarities with TMDCs. The first quasi-2D perovskite  (RNH$_3$)$_2$MA$_{n-1}$Pb$_n$I$_{3n+1}$ (n = 1 to $\infty$) was obtained as an intermediate compound between MAPbI$_3$ (n = $\infty$) and (RNH$_3$)$_2$PbI$_4$ (n = 1)~\cite{calabrese1991preparation}. The use of bulky organic cations instead of traditional small cations breaks the original continuous three-dimensional structure and creates a stable quasi-two-dimensional geometry. In fact, quasi-2D perovskites demonstrate unique optical properties due to their structural characteristics, which are different from those of bulk crystals~\cite{saidaminov2015high}. Quasi-2D perovskites are composed of a series of alternating inorganic and organic layers. The inorganic layers are located between the layers of large organic spacers with relatively low dielectric constants. Therefore, inorganic layers act as quantum wells, whereas organic coating layers act as barriers. Consequently, the charge carrier confinement potentials in quasi-2D perovskite are formed naturally with an atomically sharp interface between the barriers and wells. This organisation results in quantum-confinement effects which causes larger exciton binding energy compared to the bulk analogues~\cite{cheng2020multiple}. Consequently, such materials are characterized by increased radiative recombination rate and photoluminescence quantum yield~\cite{wang2016perovskite}. Additionally, the charge carrier confinement in quasi-2D perovskites is dependent on the thickness of quantum wells determined by the number of inorganic layers, which allows tuning their emission spectra~\cite{quan2018perovskites} across the whole optical and near-infrared spectral ranges~\cite{chu2020large,he2019high,wu2019alternative}. Consequently, quasi-2D perovskites show great promise for applications in optoelectronics~\cite{yang2018oriented, fu2019tailoring, tonkaev2021acceleration}.

%nanowires

The next natural step on the path to increasing the quantum confinement is the creation of nanowires. The first demonstration colloidally synthesized halide perovskite nanowires in solution phase without any catalyst at room
temperature was done by Zhang et al.\cite{zhang2015solution} Tuning of the emission wavelength available for perovskites \textit{via} anion exchange  enables multicolor nanowire heterojunctions, which have been demonstrated using CsPbBr$_3$ (X = Cl, Br, and I) using mixed halide compositions\cite{dou2017spatially}. Due to strong optical mode confinement, such structures are perspective as microlasers. The first demonstration of lasing in perovskite nanowires was observed in CH$_3$NH$_3$PbX$_3$ structures \cite{zhu2015lead}. Although these hybrid perovskite nanowires showed excellent lasing performance, a critical limitation of their instability motivated the replacement of the reactive CH$_3$NH$_3$ cation with an alternative cation. Consequently, Cs-based all-inorganic halide perovskites have emerged to replace the organic components  by Cs for improved chemical stability \cite{pushkarev2018few,eaton2016lasing}.
However, for achievement of quantum confinement in nanowires it is necessary to apply alternative fabrication methods that provide much thinner structures. In Ref.~\citenum{zhang2019increasing}, porous alumina membranes were used as templates  for fabrication of arrays of  vertical CH$_3$NH$_3$PbI$_3$ nanowires with controllable and uniform diameters. Particularly, nanowires with a diameter of 5.7 nm demonstrate a 56-fold enhancement in photoluminescence quantum yield and a 2.3-fold increase in light out-coupling compared to bulk nanowires. Such kind of structures were applied for creation of ultra-fast memristors \cite{poddar2021down} and effective light-emitting diodes\cite{zhang2020three}

% QD

In general, even stronger quantum confinement can be achieved in quantum dots. Colloidal quantum dots made of conventional semiconductors have shown their high photoluminescence quantum yield and narrow emission bandwidth~\cite{shirasaki2013emergence}. Perovskite quantum dots have appeared as colloidal nanocrystals (NCs), which are prospective for both fundamental research studies and practical applications due to their outstanding optical and chemical properties~\cite{shamsi2019metal,dey2021state}. Perovskite NCs have variable emission wavelengths, which are tuned by the size due to quantum confinement effect. This effect takes place when the wave functions of electrons and holes are spatially restricted to dimensions smaller than the Bohr radius~\cite{sercel2019exciton}. In CsPbBr$_3$ NCs, strong quantum confinement appears only in the smallest NCs with edge length of the order of 4~nm~\cite{bucher2019tailoring} when band gap is 2.7~eV. The corresponding band gap changes from 2.37 to 2.50~eV are caused by decreasing particle size from 7.3 to 4.1~nm. For the sizes larger than 10~nm (i.e., much larger than the Bohr radius for exciton), spatial dispersion of excitons makes an additional contribution to the emission line spectral shift~\cite{berestennikov2019beyond}. There are several methods of perovskite NCs synthesis, which allows to control the NCs size~\cite{zhang2015brightly,lignos2016synthesis, hou2017synthesis, protesescu2015nanocrystals}. The first demonstration of all-inorganic monodisperse CsPbX$_3$ NCs with high photoluminescence quantum yield of 90\% was done by Protesescu \textit{et al.} using the hot-injection synthesis methodology~\cite{protesescu2015nanocrystals}. The emission spectra of these NCs were tunable across the range of 400–700~nm (with emission linewidths of 12–42~nm) by composition change and quantum size-effect. CsPbX$_3$ NCs demonstrate improved optical properties and  chemical durability, which makes them attractive for application in optoelectronics. Moreover, colloidal perovskite NC can be readily processed in the solution, which allows fast, large-area, and cost-effective production of emitting layers, and provides compatibility with flexible electronics~\cite{wang2018perovskite1, zhao2017highly}. Thus, the structural and chemical diversity of perovskite NCs contributes to a wide range of optoelectronic applications\cite{kovalenko2017properties}.

% Superlattices

As noted above, colloidal NCs of cesium lead halide perovskites can be synthesized with narrow size dispersion. These high quality NCs can be packed in ordered arrays, the so-called superlattices, by means of solvent-drying-induced spontaneous assembly\cite{boles2016self, geuchies2016situ}. It was shown that superlattices assembled from perovskite NCs support superfluorescence \cite{raino2018superfluorescence}. Superfluorescence is a many-body quantum phenomenon caused by the coupling of emitters ensemble \textit{via} a common light field. In the case when the excited emitters are initially fully uncorrelated, the coherence can be established only through spontaneously triggered correlations due to quantum fluctuations rather than by coherent excitation. The high oscillator strength and the ability to create uniform emitters make perovskite NC a perfect candidate for this role. Furthermore, these structures function as the lasing cavity and the gain medium, enabling lasing with a low threshold \cite{zhou2021quantum}. Based on cubic CsPbBr$_3$ NCs, spherical Fe$_3$O$_4$ or NaGdF$_4$ nanocrystals and truncated-cuboid PbS nanocrystals, perovskite-type binary and ternary NC superlattices were obtained \cite{cherniukh2021perovskite}. These superlattices demonstrated  super-fluorescence as well. 



\textbf{Spatial composition separation.}
As it was mentioned before, the ability to tune the band gap by varying the halide ratios allows the synthesis of mixed halide perovskites with tailored absorption and emission across the entire visible spectrum. Due to ionic nature of perovskite, mixed halide films exposed to light or electric field demonstrate phase segregation to form Br-rich and I-rich sites (Figure \ref{fig:pero1}d). Originally, ion migration in metal halide perovskite was observed in solar cells, which exhibited huge low-frequency dielectric constant\cite{juarez2014photoinduced} and hysteresis-type current–voltage behavior \cite{jeon2014solvent}. Properties of ions in perovskites have been studied intensively recently \cite{yang2015significance}. Delocalization and diffusion of the ions are caused by weak ionic and hydrogen bonds \cite{wang2019stabilizing}. Due to low activation energy of ion migration in perovskites, directional movement of ions is easy activated by external stimuli such as light, heat or electric field. Moreover, different types of ions have different diffusion pathways, which can be modified by the presence of defects. For instance, halide vacancies and Pb$^{2+}$ vacancies migrate along the octahedron edge and the diagonal of the cubic unit cell, respectively, whereas MA$^+$ vacancies migrate toward a neighboring vacant cage~\cite{eames2015ionic}. Thus, perovskites have an additional opportunity to be controlled by external electric field or light illumination, which can be extremely strong (e.g. absorption coefficient can be changed by orders of magnitude) taking longer than 10$^{-3}$--10$^1$~seconds time-scale because of slow character of heavy ions movement in crystal.

%\textbf{Meta-structural change}.
%The creation of a single nanoparticle allows the addition manipulation of perovskite properties by supporting Mie-resonances. In these particles photoluminescence enhanced at certain wavelength which is tuned by change of nanoparticle size \cite{tiguntseva2018light}. Moreover, some other effects, such as second harmonic generation, Raman scattering, amplified spontaneous emission \cite{tiguntseva2020room} and optical cooling \cite{tonkaev2019optical} can be significantly enhanced. The structure assembled by several ordered particle maintains collective mode. This mode of the nano-oligomers opens up new possibilities for light control. Moreover, the creation of ordered arrays of perovskite nanoparticles enable unique opportunities.


\subsection{Perovskite metaphotonic designs}



\begin{figure}
    \centering
    \includegraphics[scale=0.7]{Figs/fig2_perovskite_v2.pdf}
    \caption{
    %\textbf{by TONKAEV.} 
      Applications of perovskite particles and metasurfaces. (a) The concept of gas sensor based on Fano resonant nanoparticle. Adopted from Ref.~\citenum{tiguntseva2018tunable} Copyright 2018 American Chemical Society. (b) SEM image of a cubic perovskite nanoparticle with a linear size of 310~nm supporting third-order Mie resonance and its lasing spectrum. Adapted with permission from Ref.~\citenum{tiguntseva2020room} Copyright 2020 American Chemical Society. (c) CIE 1931 color diagram with shown colors of nanocubes with the sizes in the range of 310-560~nm. Adapted with permission from Ref.~\citenum{tiguntseva2020room} Copyright 2020 American Chemical Society. (d) The illustration of an ultrafast nanomodulator based on Fano resonance. Adapted with permission from Ref.\citenum{franceschini2020tuning} Copyright 2020 American Chemical Society.
      (e) Coloration of metasurfaces assembled from particles of different size. Adapter with permission from \citenum{gholipour2017organometallic} Copyright 2017 Wiley-VCH. 
    (f) Holographic image “HIT” in with “ON” and “OFF” states corresponding to MAPbBr$_3$ perovskite metasurface and converted MAPbI$_3$ perovskite metasurface. Adapted with permission from Ref.\citenum{zhang2019lead} Copyright 2019 WILEY - VCH VERLAG GMBH \& CO .
    (g) Infrared converters based on perovskite metasurface enhancing multiphoton photoluminescence. Adapted with permission from~Ref.\citenum{fan2021enhanced} Copyright 2021 American Chemical Society.
    (h) Vortex microlaser based on perovskite metasurface. Adapted with permission from Ref.~\citenum{huang2020ultrafast} Copyright 2020 The American Association for the Advancement of Science.
    (i) The concept of the topological metasurface functionalized with perovskite nanocrystals. Adapted with permission from  Ref.~\citenum{berestennikov2021enhanced} Copyright 2021 American Chemical Society.
   (j) Perovskite metasurface-based photodetector. Adapted with permission from~Ref.~\citenum{wang2016nanoimprinted} Copyright 2019 American Chemical Society. }
    \label{fig:pero2l}
\end{figure}


\textbf{Perovskite meta-atoms.}
Due to the relatively high refractive index of halide perovskites (see Figure \ref{fig:property}b), it is possible to utilize them for the creation of nanoparticles that support Mie resonances. The resulting enhancement of optical field inside the nanoparticle further augments the exceptional properties of perovskites. %Thus the already outstanding properties of perovskite can be additionally enhanced in particle. 
In particular, this approach was used to enhance the photoluminescence \cite{tiguntseva2018light}, second harmonic generation, Raman scattering, amplified spontaneous emission \cite{tiguntseva2020room} and optical cooling \cite{tonkaev2019optical}. Importantly, perovskite nanoparticles can be produced by relatively simple fabrication methods, such as laser ablation \cite{tiguntseva2018light, tiguntseva2018tunable} and chemical synthesis \cite{tiguntseva2020room} and do not require multi-stage lithography processes.
 

%perovskite Fano
One of the first demonstrations of Mie-resonant perovskite nanoparticles was achieved using laser ablation~\cite{tiguntseva2018light}. The quasi-spherical perovskite nanoparticle was studied as an active optical nanoantenna which allows for a significant enhancement of the photoluminescence and control of the emission directivity. Another aspect of the interaction of light with perovskite nanoparticles arises due to the pronounced exciton resonance of perovskite at room temperature~\cite{palmieri2020mahan}. When a narrow exciton resonance couples to a broader cavity resonant mode, interference effects, manifested as the characteristic Fano lineshape in the spectrum, are expected. The fact that the bandgap and, consequently, the exciton wavelength of perovskites can be reversibly tuned by the anion exchange opens a novel way for controlling the optical response of nanoparticles as it was demonstrated in the work by Tiguntseva \textit{et al.}\cite{tiguntseva2018tunable} (Figure~\ref{fig:pero2l}a). There, the Fano resonance arising due to interference of the exciton with one of the Mie modes of the nanoparticle was manifested as an asymmetric dip in the scattering spectrum. In the same work, it was shown that Fano resonance can be chemically tuned across a 100~nm bandwidth in the visible range, which can be employed in the development of the optical gas sensors. 

%perovskite nanolasers and nanopixels
Despite the significant increase in photoluminescence demonstrated for laser-ablated nanoparticles, their imperfect shape limits the ability to use them as nanocavities for lasing. Conveniently, chemical synthesis allows to obtain perfectly shaped monocrystalline cubic CsPbBr$_3$ nanoparticles. This approach allowed creating an all-dielectric nanolaser with a size of 0.58$\lambda$ functioning at room temperature \cite{tiguntseva2020room}. The nanocube laser with a size of 310~nm was chemically grown on a sapphire substrate, which ensured low concentration of the defects and decreased the lasing threshold. It was revealed from the numerical modeling that the lasing was achieved at the wavelength of a 3rd order Mie-mode (Figure~\ref{fig:pero2l}b).

In addition, perovskite nanoparticles supporting low-order Mie resonances in the visible range can be used for color generation in scattered light similarly to that shown with other materials~\cite{kristensen2016plasmonic}. In Ref.~\citenum{tiguntseva2020room} it was shown that cubic nanoparticles exhibit a strong dependence of the color of the scattered light on the particle size (Figure~\ref{fig:pero2l}c) and cover the ranges from yellowish to reddish. Suppression of the blue color appears due to high material adsorption in this range. Nonetheless, the range of operating colors can be extended by the halide anion exchange.

%nanomodulators
Another example of the employment of Fano resonances formed by the coupling of an excitonic state with Mie modes in halide perovskite is photo-induced subpicosecond optical modulator that was proposed in Ref.~\citenum{franceschini2020tuning} (Figure~\ref{fig:pero2l}d). There, it was revealed that ultrafast modification of the optical properties induced by the band gap renormalization and band filling mechanisms significantly depends on the geometry of the chemically produced nanoparticle. When the nanoparticle size is chosen to achieve an overlap between narrow-band excitons and geometry-driven Mie resonances, the ultrafast modulation of the transmittance can be completely opposite for the nanoparticles of different sizes. For example, 150-nm quasi-spherical CsPbBr$_3$ nanoparticle where Mie-exciton coupling is weak exhibits up to +1\% of transmittance modulation on picosecond scale, while 300-nm nanoparticle with pronounced Fano resonance exhibits up to -1\% modulation. Thus, Mie-exciton coupling offers an additional parameter to control nanoantennas and ultrafast optical switches.


\textbf{Perovskite metasurfaces.}
Despite the wide range of applications for single nanoparticles, the creation of a metasurface from meta-atoms can significantly expand their scope. However, due to the random distribution of nanoparticle size and location when using laser printing and various chemical synthesis methods, it is difficult to create highly ordered arrays of nanoparticles. Thereby, methods such as nanoimprint lithography~\cite{makarov2017multifold, wang2016nanoimprinted}, electron beam lithography~\cite{fan2021enhanced, zhang2019lead}, laser projection lithography~\cite{zhizhchenko2020light, zhizhchenko2021direct}, and focused ion beam lithography \cite{baryshnikova2020broadband, gholipour2017organometallic}, are engaging for the fabrication of metasurfaces based on halide perovskites. 

%structural coloring
One of the first applications of perovskite metasurfaces was structural coloration~\citep{gholipour2017organometallic} enabled by Mie resonances. As shown in Fig.\ref{fig:pero2l}e, perovskite metasurface with certain size of meta-atoms demonstrates vivid and tunable color generation in reflected light. Moreover, as we discussed in the previous paragraphs, in contrast with the most of standard dielectric materials, halide perovskites can be reversibly tuned chemically, which can be applied for dynamic tuning of the structural colors of metasurface. 
%tunable pixels
Such a novel approach for reversible \textit{in situ} color tuning in a perovskite metasurface was firstly demonstrated by Gao et al~\cite{gao2018lead}.

Then, in Ref.~\citenum{zhang2019lead}, Zhang \textit{et al.} demonstrated the holographic image created by MAPbBr$_3$ metasurface, which can be switched off with the anion exchange in a CVD tube. Figure~\ref{fig:pero2l}f shows this reversible transition between “ON” and “OFF” status of the perovskite metasurface by the anion exchange turning MAPbBr$_3$ perovskite to MAPbI$_3$, which led to significant enhancement of absorption at the excitation wavelength, whereas the refractive index remained almost unchanged. Consequently, the phase shift was preserved while the intensity decreased. The reversibility of the effect paves the way for development of the tunable pixels based on perovskite metasurfaces.

%IR conververtrs 

Lead halide perovskites also have strong nonlinear absorption\cite{walters2015two,xu2020halide}, which makes them prospective for converting the infrared radiation into the visible range emission. The application of the concept of metamaterials allows to significantly increase the two-photon photoluminescence (Figure~\ref{fig:pero2l}g). Perovskite metasurfaces created by the application of nanoimprint lithography demonstrated the enhancement of nonlinear photoluminescence intensity up to 70 times compared to a pristine perovskite film~\cite{makarov2017multifold}. This achievement became possible due to the enhancement of electric field in the structure driven by magnetic Mie resonances at the excitation wavelength. Moreover, an additional contribution to the radiation enhancement is provided by another Mie mode at the emission wavelength, which also results in the acceleration of photoluminescence due to the Purcell effect. 

By the use of a more precise technique as electron beam lithography, the higher quality of the structure is achieved. Perovskite metasurface fabricated with the application of this method demonstrates outstanding enhancement of two-photon photoluminescence as its intensity becomes comparable with a linear one~\cite{fan2021enhanced}. Moreover, the threshold of amplified spontaneous emission for two-photon pumping is only 2.7 times higher than that for the linear case. Additionally, by engineering the metasurface, the enhancement of nonlinear luminescence is possible even for three- and four-photon excitation. Interestingly, the resonantly enhanced nonlinear photoluminescence from perovskite metasurfaces was also employed for optical information encoding~\cite{fan2019resonance}.

%microlasers

As we discussed previously, halide perovskites is one of the best families of materials for application in compact lasers. The perovskite-based microlasers were demonstrated based on different photonic designs including nanowires, microdisks, and gratings~\cite{zhu2015lead,zhang2017advances,wei2019recent,polushkin2020single,qin2020stable}. Recently, basing on the BIC platform~\cite{kodigala2017lasing}, the novel conception of etchless perovskite microlasers was demonstrated with perovskites~\cite{wang2021highly}. It was shown that the BIC mode can be supported by the top polymer grating without the harmful chemical etching of the perovskite film. Thereby, the lasing properties of perovskite are well preserved in such structures. Also, perovskite metasurfaces supporting BIC mode were developed as vortex microlasers with relatively low threshold~\cite{huang2020ultrafast}. Moreover, these perovskite-based vortex microlasers allowed for all-optical switching at room temperature (Figure~\ref{fig:pero2l}h). Moreover, due to the fact that transition from BIC vortex lasers to linear lasers represented a redistribution of the laser emission instead of a direct switching of the lasing mode with the switching time about 1 to 1.5 picoseconds that is several orders of magnitude lower than it was demonstrated before. Such devices can be useful for high-speed classical quantum communication~\cite{qiu2017vortex}.

%Chiral metasurfaces

Regarding the applications related to rotation of light polarization, there is still a challenge to synthesize hybrid perovskites with large optical chirality due to the relatively limited degree of structural twisting and, thus, weak chirality transfer from chiral molecules to the perovskite framework, optical activity, and distinguishable circularly polarization imparted on light passing through the sample~\cite{long2020chiral}. In the work~\cite{long2022perovskite}, all-dielectric perovskite metasurfaces with giant super-structural chirality was realized. The authors optimized the electric- and magnetic-type resonances of the resonant chiral meta-molecules to obtain large anisotropy factor of 0.49 and circular dichroism of 6350 mdeg. However, theoretically such larger area metasurfaces could potentially exhibit even higher optical activity, approaching the theoretical limits ($<$45~000~mdeg).

%topology

Application of specifically designed metasurfaces allows to reach topological states in visible range~\cite{lu2014topological}, which is prospective even for lasing applications~\cite{bandres2018topological,zeng2020electrically}. On this way, halide perovskite nanocrystals were also integrated with silicon-based kagome lattice supporting various topologically protected states~\cite{berestennikov2021enhanced} (Figure~\ref{fig:pero2l}i). By the anion exchange reaction the emission wavelength was precisely tuned to the topological band of the structure at 650~nm which caused a significant enhancement at the frequency of zero-dimensional topological corner states. The use of the topology resulted in bosonic polariton condensation in a one-dimensional strong lead halide perovskite lattice at room temperature~\cite{su2020observation}. Further development this idea with application of zigzag chain make it possible to switch between distinct topological phases by polarization change~\cite{su2021optical}. The employment of BIC concept allows to extend the polarisation control. Thus topological polarization singularities in polariton emission was demonstrated~\cite{kim2021topological}. In this structure a very large degree of circular polarization was achieved. In addition, by nanograting of perovskite nanowire, a topological defect into the nanowire can be introduced and a defect-localized lasing state is observed~\cite{berestennikov2020optical}.

%modulators

Perovskites have the relatively high photoconductivity and optical properties which can lead to realisation of active devices based on perovskite-metasurface system~\cite{manjappa2017hybrid}. Therefore, hybrid lead halide perovskites could be a promising candidate for active photo-induced modulation. It was exhibited that  hybrid perovskite–metasurface device shows extremely low power photoswitching of the metasurface resonances in the terahertz part of the electromagnetic spectrum \textit{via} Fano resonance. Moreover, a typical coupled photon–metasurface resonance is observed at higher pump powers, where the Fano resonance amplitude is extremely weak. In addition, the observed outstanding properties of the system with ultrasensitive switching could offer the new generation of photonic metadevices with significant benefits. Also, acceleration of perovskite optical response can be achieved by integration with hyperbolic metamaterials because of Purcell factor on their surfaces~\cite{adl2020purcell, li2020active, tonkaev2021acceleration}.


%antireflection coating

In smart-windows, AR and VR application, it is tempting to use semi-transparent light-emitting devices with antireflective properties. In this regard, very prospective approach was recently demonstrated, which is based on a perovskite metasurface representing nanoparticles with bullet-like shape decreasing the reflectance from 33\% to 4\% and supporting 15\% enhancement of the photoluminescence compared with pristine film~\citep{baryshnikova2020broadband}. Taking into account three times less volume of perovskite material in these nanoparticles as compared with the film, it can be concluded that the photoluminescence yield was significantly enhanced.


\subsection{Perovskite devices improved with metaphotonics}

\textbf{Photovoltaics.}
For solar cells applications, halide perovskites can be considered both as primary and functionalizing materials in metaphotonic designs. In the work~\cite{wang2018diffraction}, direct nanostructuring of a perovskite photoactive layer in solar cell was demonstrated to be useful for power conversion efficiency and photocurrent density enhancement from 16.71\% and 21.67 mA~cm$^{-2}$ to 19.71\% and 23.11 mA~cm$^{-2}$, respectively. The main mechanism responsible for such improvement was proved to be light trapping. Another approach for perovskite solar sells improvement is incorporation of optically Mie-resonant high-index NPs into the perovskite layer in the device, which can also provide light absorption in the whole working spectral range~\cite{furasova2018resonant}. Relatively low losses in silicon (at wavelengths $>$500~nm) and pronounced Mie resonances result in not only local field enhancement but also increase of the radiative Purcell factor. Thus, the external quantum yield of a perovskite film increases due to the presence of the silicon NPs, which increase probability of the photon recycling (because the re-emitted photons are absorbed in the same area) and makes parasitic non-radiative carriers recombination on defects less efficient~\cite{dequilettes2019charge}. As a result, both photo current and the open-circuit voltage grow~\cite{furasova2018resonant}. Remarkably, that Mie-resonant Si NPs can be additionally optimized for the photovoltaic purposes by its doping~\cite{furasova2020engineering} or positioning in right places of the charge carriers transport layers~\cite{furasova2021mie} of the device. Such improvements of conversion efficiency up to 21.1\% with high-index dielectric NPs~\cite{furasova2021mie} turned out to be usually better as than that with plasmonic NPs (record efficiency was 20.1\%~\cite{zhao2020enhanced}) because of higher optical losses in metals in visible range reducing the voltage of a solar cell. In order to achieve and overcome the record high conversion efficiencies for perovskite solar cells ($\sim$25\%~\cite{NREL}) by means of nanophotonic approaches, it is important to combine them with the most advanced perovskite compositions (e.g. like triple-cation ones~\cite{saliba2016cesium}) and care about defects passivation~\cite{gao2020recent}.

\textbf{Photodetectors.}
The photoresponsivity and response time of perovskite photodetectors are between TMDs and black phosphorus~\cite{wang2021recent}. Optimization of optical properties by smart nanostructuring can be used to improve the photoresponse and suppress the dark current. The first nanophotonic optimization of a perovskite photodetector was demonstrated with Au nanocrystals~\cite{dong2016improving}, where a 238\% plasmonic enhancement factor and 10$^6$ on/off ratio was demonstrated. Various plasmonic designs based on Au, Ag and Al nanoparticles were also proposed for perovskite photodetectors~\cite{sun2016plasmonic,wang2017plasmon,liu2019using,wang2018perovskite,li2018plasmonic, li2020enhanced, wang2020boosting}, where one of the highest nanostructure-empowered enhancement of external quantum efficiency as high as 2.5 times was demonstrated in the work~\cite{du2018plasmonic}. Also, near‐infrared circularly polarized light detectors fabricated using chiral Au NPs embedded to Pb-Sn perovskite films exhibited remarkable chirality in the near-infrared region with high anisotropic response, owing to a giant plasmon resonance shift of chiral plasmonic Au NPs~\cite{kim2022ultrasensitive}, which might be prospective to chiral perovskite optoelectronics~\cite{long2021all,long2020chiral}.

Non-plasmonic metaphotonic designs are also extremely promising for photodetection improvement. 1D grating was applied for planar phtotodetector architecture providing approximately 35 times improvement on responsivity and 7 times improvement on on/off ratio compared with the nonimprinted devices~\cite{wang2016nanoimprinted} (Figure \ref{fig:pero2l}j). Similar nanophotonic design with period around 1~$\mu$m imprinted on CsPbBr$_3$ layer was shown to be a prospective design to improve sensitivity of self-powered perovskite photodetector with extended spectral range from UV up to near-IR~\cite{cao2019self}. Nanoimprint technology also allows for fabrication of large scale 2D surface gratings mimicking butterfly wings for application in perovskite photodetectors. Such type of designs provide 7 times enhancement of responsivity and detectivity as compared with a pristine perovskite photodetectors~\cite{zhan2019butterfly}. Remarkably, polarisation sensitivity can be very high in such grating-based metaphotonic designs. In the work~\cite{jing2020hybrid}, 2D surface grating perovskite array of Mie-resonant nanoparticles fabricated by focused-ion-beam nanolithography showed the increasing of photocurrent up to $\sim$40\%, the related internal efficiency by $\sim$20\% compared to the flat film of the same material.

\textbf{Light-emitting devices.}
Generally, in perfectly planar LEDs up to 80\% of the generated light is trapped in the device, which lead to increasing of non-radiative losses and limits opportunity for efficiency improvement~\cite{lee2003high}. Several methods, such as diffraction gratings and low-index grids, have been used to increase the out-coupling of the light trapped in perovskite LEDs~\citep{richter2016enhancing}. However, the application of these techniques involve complicated fabrication and lead to distortions in the spectrum and emission directivity~\citep{ziebarth2004extracting}. It was revealed that solution-processed MAPbI$_3$ perovskites that spontaneously form a layer of random nanoparticles can efficiently extract generated light from the devices~\cite{cao2018perovskite}. Moreover, this perovskite device retains wavelength and directivity of electroluminescence. 

An additional efficiency growth for nanostructured devices can be achieved due to passivation of surface defect and decrease of non-radiative recombination. This approach was demonstrated as perspective for the development of high-brightness standard perovskite LEDs with efficiency exceeding 20\% level~\cite{xu2019rational,ye2021defect}.


\section{Solution-based functional platforms}

\subsection{Tunable meta-shell nanoparticles}

As mentioned above, optical resonances play a crucial role in dielectric metaphotonics, because they allow to enhance substantially both electric and magnetic responses, which provide unique opportunities for light manipulation at subwavelength scale. On the other hand, for many applications of Mie-resonant nanoparticles, it is important to achieve tunability and stimuli-responsive functionality creating bio-integrated optical nanostructures. With this motivation in mind, Nikitina et al~\cite{nikitina2021all} covered silicon nanoparticles by an ensemble of similarity negatively charged polyelectrolytes (heparin and sodium polystyrene sulfonate). They created thermo-responsible system based on Si nanoparticles and a shell produced by poly-(diallyldimethylammonium chloride) (PDADMAC) and poly-(sodium 4-styrenesulfonate) (PSS) assembly. The dynamic tuning of the optical response of the resulting meta-shell nanoparticles (see Fig.~\ref{fig:yuri1}) was achieved by their light-induced heating and swelling of the polyelectrolyte shell. As clearly seen from Figs.~\ref{fig:yuri1}(a,b), the resulting hydrophilic/hydrophobic transitions significantly change the shell thickness and reversible shift of the scattering spectra for individual nanoparticles up to 60 nm, as shown in Fig~\ref{fig:yuri1}(c). Thus, it was revealed that the hydrophilic/hydrophobic transitions may lead to a significant change in the shell thickness and reversible shift of scattering spectra at mild conditions.

\begin{figure}[h]
    \centering
    \includegraphics[scale=0.9]{Figs/fig1_MSNP.pdf}
    \caption{. (a) Schematic of the thermal transformations of a meta-shell silicon nanoparticle covered by poly-(sodium 4-styrenesulfonate) (PSS) foam with the contribution of negatively charged heparin that provides hydrophilic repulsive and hydrophobic attractive forces. (b) Bright-field transmission electron microscopy (TEM) images of a meta-shell system before and after irradiation, respectively.  (c) Scattering spectra for the single meta-shell silicon nanoparticle with thermoresponsive polymers shell (Adapted with permission from Ref.\citenum{nikitina2021all} Copyright 2021 Angewandte Chemie).}
    \label{fig:yuri1}
\end{figure}



The multipolar expansion performed in Ref.\citenum{nikitina2021all} demonstrates that the broadband electric dipole resonance causes a significant contribution to the scattering spectra shift. Unlike its magnetic
counterpart, the electric dipole’s electric field penetrates the polyelectrolyte shell layer, which is tuned practically under laser illumination. This effect may allow the resonance overlapping and scattering spectrum reshaping.

Subwavelength dimensions of the building blocks enable engraving optical metamaterials within meta-shell particles,  leading to effective optical features in a colloidal platform with ability to tune the electromagnetic responses of these particles.  Bahng et al~\cite{bahng2020mie} demonstrated meta-shell nanophotonics with dielectric colloidal superstructure having an optical nonlinear metamaterial shell conformed onto a spherical core. We demonstrated that the metamaterial shell facilitates engineering the Mie resonances in the nanoparticle that enable significant enhancement of the second-harmonic generation (SHG) with the several orders of magnitude enhancement compared to its building blocks. 

\begin{figure}[h]
    \centering
    \includegraphics[scale=0.9]{Figs/fig2_MSNP.pdf}
    \caption{Nonlinear nanophotonics with colloidal meta-shell nanoparticles. (a) The nanoparticle, of diameter d, is synthesized by assembling ZnO nanorods, of width w and length l, into a spherical array, forming a meta-shell conformed onto a dielectric $\mu$-sphere. (b) Scanning electron microscopy images of the meta-shell particle synthesized with a polystyrene $\mu$-sphere core. (c) Second harmonic generation of light by the meta-shell nanoparticle: experimental image and an artist impression. Adapted with permission from Ref.~\citenum{bahng2020mie} Copyright 2020 American Chemical Society.}
    \label{fig:yuri2}
\end{figure}

The authors of Ref.\citenum{bahng2020mie} demonstrated conversion efficiency as high as 10$^{-7}$ far from the damage threshold, setting a benchmark for SHG with low-index colloids. The meta-shell nanoparticles provides pragmatic solutions for instantaneous wavelength conversions with colloidal platforms that are suitable for chemical and biological applications. This meta-shell approach for enhancing optical nonlinear processes in nanostructures is in sharp contrast with the ongoing nonlinear nanophotonic efforts. While most of the efforts utilize high Q-factor Mie resonances confined within the structures usually comprised of high-index materials, these meta-shell approach utilizes collective interferences of high-density low-Q-factor modes in low-index materials, which lead to formation of strong hotspots that is utilized to enhance the conversion efficiency. Furthermore, the metamaterial shell allows for engineering the Mie resonances  to maximize the spatial overlap of the hotspot with the regions having highest  density of the quadratic colloidal elements. The presented platform utilizes the low-index wide-bandgap colloidal nanoparticles made of ZnO for nonlinear optical wavelength conversion. While ZnO does not have a large second-order susceptibility $\chi^{(2)}$ compared to other commonly used nonlinear materials, its well-developed chemical synthesis in the colloidal platforms makes it compatible with a wide array of scientific and industrial processes. Furthermore, the demonstrated concept
can be extended to metallic structures to utilize plasmonic resonances to further expand the possibilities of nonlinear nanophotonics with colloidal platforms. Combining chemical and optical properties that arise from the  meta-shell supraparticle (MSP)  packaged into a single nanoparticle could further expand development of the colloidal nonlinear nanophotonics to chemistry and biology.

\subsection{Upconversion nanoparticles}

%what is upconversion
Novel prospective type of nanomaterials for integration with advanced metaphotonic designs is upconversion nanoparticles (UCNPs)~\cite{zhou2015controlling,marin2020doping}, representing host dielectric particles doped with trivalent lanthanide ions (Ln$^{3+}$) listed in Figure~\ref{fig:upcon1}a.
Indeed, the rare-earth-doped nanocrystals exhibiting upconversion properties are emerging light sources used for many applications of nanotechnology
such as sensing, bioimaging, therapy, display, data storage, and photovoltaics. 
Their main functionality is related to efficient conversion of invisible incident IR photons to visible light. Because the both nonlinear harmonics generation and photoluminescence driven by multiphoton absorption are the nonlinear processes occurring through virtual levels in the material band gap rather than through real states, their efficiency is still relatively low in metaphotonic designs.
Unlike two-photon absorption and second-harmonic generation, photon upconversion deals with sequential, rather than simultaneous, absorption of incident photons. This nonlinear optical phenomenon results in an anti-Stokes emission, where two or more low-energy photons are converted to high-energy luminescent emission. 

\begin{figure}[b!]
    \centering
    \includegraphics[width=0.95\linewidth]{Figs/UpCon1.pdf}
    \caption{
    %\textbf{by MAKAROV.} 
    (a) Ln$^{3+}$ ions used for upconversion and typical Ln$^{3+}$-based upconversion emission bands covering a broad range of wavelengths from ultraviolet to NIR and their corresponding main optical transitions. (b) Simplified energy level diagrams depicting main upconversion mechanisms including excited-state absorption (ESA), energy transfer upconversion (ETU), cooperative sensitization upconversion (CSU), and photon avalanche (PA). (c) Typical types of upconversion nanoparticles~\cite{liu2013lanthanide, fan2019exploiting}. (d) Principles of manipulation by the luminescence of lanthanide-doped UCNPs.}
    \label{fig:upcon1}
\end{figure}

%mechanisms
Generally, the upconversion process in Ln-doped UCNPs can be categorized as following:  excited state absorption (ESA), energy transfer upconversion (ETU), cooperative sensitization upconversion (CSU), and photon avalanche (PA)~\cite{bunzli2005taking, dong2013basic, wu2019expanding} as schematically shown in Fig.~\ref{fig:upcon1}b. ESA is the simplest approach, which however does not give high efficiencies. In turn, ETU is
known to be the most efficient upconversion mechanism, which typically
realized with a sensitizer(S)–activator(A) system. After absorbing the
excitation photons, resonant energy transfer from the sensitizer
to the activator populates the intermediate state of the activator center. As a result, the record-high values of upconversion photoluminescence quantum yield were demonstrated up to 7.6\% ~\cite{huang2014lanthanide}.
%\textbf{Electric and magnetic types of transitions.}
Remarkably, that trivalent lanthanides possess not only electric-dipole transitions but also magnetic-dipole ones, which provide additional degree of freedom for spectrally selective emission enhancement~\cite{baranov2017modifying}. The magnetic-dipole transitions in trivalent lanthanides were found in ultraviolet, visible, and near-infrared spectral ranges~\cite{dodson2012magnetic}. 

%sizes and types
Integration of the upconversion materials with metaphotonic nanostructures requires specific properties of light-emitting material. For example, the usage of bulky solid-state substrates doped with Ln$^{3+}$ is not always applicable for nanophotonics, where all optical effects happen in thin near-field layer. 
In Fig.~\ref{fig:upcon1}c one can see different types of nanoparticles, where upconversion lanthanide ions are usually placed. Upconversion nanoparticles (UCNPs) can be of the sizes from several up to hundreds nanometers~\cite{zhou2015controlling, wen2018advances}. Moreover, various advanced techniques of multi-layer core-shell UCNPs were applied to adjust absorption/emission wavelength and quantum yield of photoluminescence~\cite{fan2019exploiting}. Also, UCNPs can be dye-sensitized~\cite{wang2017dye}, integrated with organic molecules~\cite{han2020lanthanide}, or decorated by other NPs~\cite{clarke2018large}. 

%tunability
Because of complex structure of levels system in UCNP, the emission wavelength and efficiency are dependent on external parameters. Therefore, various strategies were reported to change the luminescence properties of UCNPs, including emission tuning induced by incident laser intensity, electric field, magnetic field, pressure, pH, and temperature~\cite{gonzalez2015upconversion, wu2019expanding, zhou2020single}, as shown schematically in Fig.~\ref{fig:upcon1}d.

%Electric field
Electric-field variation of Ln-doped BaTiO$_3$ thin film emission was firstly demonstrated in~\cite{hao2011electric}. The photoluminescence intensity was modulated with DC and AC electric field, providing a real-time and dynamic way to control photoluminescence. The upconversion enhancement emission was observed through the converse piezoelectric effect, where the increase in the radiative transition probabilities in Ln$^{3+}$ is caused by variation of its cite symmetry dependent on structure symmetry of the BaTiO$_3$ host affected by an electric field.
Electric-field induced temporal modulation of anti-Stokes luminescence can be also achieved by coupling UCNPs with an electrochemically responsive molecule~\cite{wu2021dynamic}. In such a system, dynamic colour editing of anti-Stokes luminescence at single-particle resolution was achieved by electrically tailoring orbital energy levels of the molecules surrounding the UCNP surface. With this technological platform, it becomes possible to convert information-encrypted electrical signals into visible patterns with millisecond photonic readout.

%Magnetic field
Similar strategy to tune upconversion luminescence can be realized by introducing a magnetic field, which modify the energy levels of the Er$^{3+}$ ions. The magnetic mechanism of the green emission reduction at the fields below 1~Tesla is due to the Zeeman splitting of energy levels of the Er$^{3+}$ ions was demonstrated in the system NaGdF$_4$:Nd,Yb,Er~\cite{liu2013magnetic}. On the other hand, upconversion emission enhancement can be also achieved in UCNPs with different level structure at magnetic fields as high as 30~Tesla~\cite{xiao2016dynamically}.

%Temperature, pH, pressure:
In such field as biophotonics, emission tunability of UCNPs is very prospective for local material changes in the environment, bringing critical information about the studied object. For examples, pH-sensitive~\cite{tsai2019upconversion}, temperature-sensitive ~\cite{fernandez2018continuous}, and pressure-sensitive~\cite{wisser2015strain} UCNPs were successfully demonstrated experimentally. Also, rare-earth emission was shown to be very sensitive to crystalline state of the matrix~\cite{larin2021luminescent}.

Despite tremendous progress in UCNPs optimization and dynamical control of the emission properties, they still possess relatively low quantum yield of anti-Stokes photoluminescence, being around few percents. Therefore, the application of metaphotonics seems to be the most prospective direction beside material composition optimization.   
Generally, there are three main process involved in the upconversion: 
\begin{itemize}
\item{light absorption}, 
\item{photoexcited carrier decay (radiative and nonradiative)},
\item{energy transfer}. 
\end{itemize}
All these processes can be affected by smart integration of the UCNPs with metaphotonic structures. 

{\bf Emission enhancement}. Previously, upconversion emission was enhanced on microstructures. For example, work~\cite{liang2019upconversion} reports on dielectric microbeads to significantly enhance the photon upconversion processes in UCNPs. By modulating the wavefront of both excitation and emission fields through dielectric superlensing effects (nanojects formation), luminescence amplification up to 5 orders of magnitude was achieved.

However, to shrink photonic designs to the nanoscale dimensions, it is crucial to optimize nanostructure geometry and material properties. The local field can be strongly enhanced by a suitably designed optically resonant nanostructures, thereby impacting the absorption cross-section of UCNPs. In the work~\cite{gong2019upconversion}, a dual-resonance all-dielectric metasurface (see Fig.~\ref{fig:upcon2}a) was proposed to enhance the signals emitted by UCNPs (NaYF$_4$:Yb,Tm). An averaged upconversion signal enhancement of around 400 times was achieved by designing the electric and magnetic dipole resonances of the metasurface, affecting not only the local field enhancement, but also the UCNPs quantum efficiency, respectively. Moreover, the metasurface allowed to optimize the collection efficiency making directional the emission of the UCNPs on the metasurface. Similar strong and spectrally broadband local field enhancement at incident wavelength was demonstrated in flexible plasmonic nanowires network was employed for UCNPs emission enhancement, which helped to demonstrate displays for fingerprint identification~\cite{xu2018broadband}.

In order to enhance magnetic-dipole transitions in lanthanide ions around metaphotonic structures, one can place them in the
node of a standing wave with a maximum of the magnetic field and a zero of the electric field~\cite{yang2011local}. Similar effect can be achieved with azimuthally polarized excitation beam with a pronounced maximum of the magnetic field and a zero of the electric field in the center~\cite{kasperczyk2015excitation}, which was demonstrated experimentally with Eu$^{3+}$-doped nanocrystals were pumped by an azimuthally polarized beam. Further development can be anticipated for integration of UCNPs with magnetic metaphotonic structures supporting both electric and magnetic hot-spots~\cite{nazir2014fano, bakker2015magnetic, feng2017isotropic, zanganeh2021anapole}. 

\begin{figure}
    \centering
    \includegraphics[width=0.75\linewidth]{Figs/UpCon2.pdf}
    \caption{
    %\textbf{by MAKAROV.} 
    (a) Schematic and SEM image of a dielectric metasurface, and the PL scan of three letters ''ZJU'' covered by UCNPs. Experimentally measured upconversion PL spectra on metasurface (red) and substrate (blue, scaled by 50 times), respectively. Adapted from Ref.~\citenum{gong2019upconversion} Copyright 2019 Royal Society of Chemistry Publishing. (b) Metasurface made of gold nanotrenches as plasmonic cavities for UCNPs. Luminescence enhancement factors of coupled UCNPs compared with samples on a glass, as a function of increasing pitch size. Upconversion luminescence spectra and corresponding schemes of UCNPs assembled within gold nanotrenches with 300~nm pitch, on a glass substrate, and on an unpatterned gold film. Adapted with permission from Ref.~\citenum{xu2021multiphoton} Copyright 2021 American Chemical Society. (c) Er$^{3+}$-doped core-shell nanoparticles deposited on a zigzag array. Polarization states for 15-nanodisk zigzag array revealing topological edge states of the zigzag array. Adapted with permission from Ref.~\citenum{tripathi2021topological} Copyright 2020 De Gruyter. (d) Schematic of a hyperbolic metamaterial covered by UCNPs, and CIE diagrams representing white color of emission. Adapted with permission from Ref.~\citenum{haider2018highly} Copyright 2018 American Chemical Society.}
    \label{fig:upcon2}
\end{figure}

{\bf Localization and Purcell effect}. For light localization, one of the most powerful approach is to use metal (or `plasmonic') nanostructures~\cite{schuller2010plasmonics, jiang2017active, kravets2018plasmonic}, because of their ability to support optical resonances at the scale of metal skin-depth (around 10-20~nm). In Figure~\ref{fig:upcon2}b, the example of plasmonic metasurfaces with ultra-narrow ($\sim$25~nm) trenches, supporting extremely sub-wavelength localization, is shown~\cite{xu2021multiphoton}. The gap-mode nanocavity confines incident excitation radiation into nanoscale photonic hotspots with extremely high field intensity, accelerating multiphoton upconversion processes. The coupling of upconversion nanoparticles to subwavelength gap-plasmon modes increased spontaneous emission rates by 3.7 times and enhanced upconversion by 5 orders of magnitude. 

Usually, strong localization of optical modes in metaphotonic structures results in increase of local density of photonic states, which accelerates carriers decay from the upper states and related to the Purcell effect~\cite{purcell1995spontaneous, akselrod2014probing}. For example, integration of UCNPs between plasmonic nanocubes and metal substrate in 20-nm gap resulted in 166-fold acceleration of emission rate, which reduced the photoluminescence time from milliseconds down to sub-2-$\mu$ range~\cite{wu2019upconversion}.
Thus, in the light-emitting systems, it becomes crucial to provide resonant optical response not only at pump frequency, but also in the spectral range of emission~\cite{park2015plasmon}. For the upconversion applications, the importance of dual-resonant optical design was confirmed by means of plasmonic nanorods, where longitudinal (at 920~nm) and transverse (at 522~nm) localized surface plasmon resonances were adjusted spectrally to the excitation and emission wavelengths of ZrO$_2$:20\%Yb$^{3+}$,2\%Er$^{3+}$@NaYF$_4$:2\%Yb$^{3+}$ UCNPs~\cite{zhan2015tens}. 

Despite the high potential of Purcell effect for UCNPs emission enhancement, there is trade-off between the positive effect of Purcell effect and the negative effect of quenching. Typically, quenching dominates at short distances and small sizes, while the Purcell effect remains pronounced at larger length and sizes~\cite{mendez2019control}. Also, the F{\"o}rster energy transfer rate acceleration nearby plasmonic nanostructure should be taken into account~\cite{sun2014plasmon, lu2014plasmon}.

{\bf Polarization control}. Another important parameter of UCNP emission is polarization. In a perfect case, UCNP emits linearly polarized light, because the polarization
anisotropy is assigned to each splitting transition of rare-earth ions, and it is determined by the site symmetry in a crystal host~\cite{zhou2013ultrasensitive, rodriguez2016determining}. However, in real systems, the irregular orientation of the crystalline axis will lead to a neutralization effect, owing to different dipole orientations for the transitions. 
To gain a control over polarization of emission, the emitters can be coupled to resonant nanostructures. Indeed, coupling of emitters to individual metaphotonics structures~\cite{curto2010unidirectional, novotny2011antennas, cotrufo2016spin}  was proposed for control over the polarization of generated light. Nevertheless, not all types of resonant nanostructures can solve the problem of polarization in UCNPs, because the resulting polarization of emission depends typically on
the specific positioning of emitters relatively to the nanoantenna or metasurface. In this regard, polarization control that relies on topologically nontrivial optical modes in nanostructures becomes attractive as topology introduces
robustness against disorder in positioning. 

In Figure~\ref{fig:upcon2}c, the example of rare-earth doped $\beta$-NaErF$_4$@NaYF$_4$ core-shell nanocrystals integration with topologically robust metaphotonic design is shown. 
 Specifically, the authors of the work~\cite{tripathi2021topological} used zigzag arrays of dielectric nanoresonators hosting topologically nontrivial optical modes that are robust against perturbations of the system~\cite{kruk2019nonlinear}. They observed that in the vicinity of topological edge states, the emission becomes not only enhanced, but also linearly polarized reproducing the polarization of topological edge modes. In particular, for the arrays with odd number of nanoresonators, the PL emission from two edges is orthogonally polarized, and for the arrays with even number of nanoresonators, the PL emission becomes co-polarized, in accordance with the polarization of the topological states. 

{\bf Upconversion lasing}. Multiphoton absorption in direct band-gap semiconductors~\cite{guzelturk2014amplified,li2015ultralow, zhou2021perovskites} \textit{via} virtual levels in band gap requires high intensities to achieve lasing, which is possible in pulsed regime of excitation only to avoid optical and thermal damage associated with intense continuous-wave (CW) excitation. In order to achieve CW upconversion lasing operation, UCNPs can be one of the best candidate. 

In order to achieve upconversion stimulated emission, it is critical to employ the most appropriate excitation mechanism allowing to provide inversion of population between upper and lower energy levels. Basing on the energy-looping excitation mechanism in Tm$^{3+}$-doped UCNPs~\cite{levy2016energy}, continuous-wave upconverted lasing action in optically resonant microcavities was demonstrated at excitation intensities as low as 14~kW$\cdot$cm$^{-2}$ for more than 5~hours at blue and near-infrared wavelengths simultaneously~\cite{fernandez2018continuous}. These microcavities are excited in the biologically transmissible second near-infrared window and are small enough to be incorporated into biological tissues or organisms. For example, the lasing microcavities were immersed in blood serum to demonstrate sensing and illumination in complex biological environments~\cite{fernandez2018continuous}. Remarkably, that even optimally doped single UCNP integrated with a cavity can provide upconversion lasing at room temperature~\cite{shang2020low}, which makes UCNPs prospective for integration with high-$Q$ metaphotonic designs.

The multicolor lasing emitted from core-shell nanoparticles covering the red, green, and blue, simultaneously, can be greatly enhanced by the high photonic density of states with a suitable design of hyperbolic meta-materials, which enables decreasing the energy consumption of photon propagation.


\section{Perspective and Outlook}

We have demonstrated how recently emerged field of metaphotonics can be empowered by adding several platforms based on exotic polymers, perovskites, transition metal dichalcogenides, and phase-change materials, for enhancing light-matter interaction and expanding towards new applications. We have reviewed different advanced techniques recently emerged in metaphotonics dealing with all-dielectric resonant nanoscale structures for applications in tunable metadevices, nanolasers, frequency conversion, as well as for emerging fields of metachemistry and ultracompact chemical and biological sensing devices. For illustration and direct comparison of the main applications, we provide Table~1 that includes the key examples of multifunctional and transformative metadevices based on the emerging materials discussed above.


\begin{table}[]
\caption{Summary on multifunctional and transformative metaphotonic designs based on emerging materials.}
\begin{tabular}{|llllll|}
\hline
\multicolumn{6}{|c|}{\textbf{Tunable metadevices and light modulators}}                                                                                                                                                                                                                                                                                                                                                                                                                                                                                                                                                      \\ \hline
\multicolumn{1}{|l|}{Application}                                                                                                       & \multicolumn{1}{l|}{{ \textbf{\begin{tabular}[c]{@{}l@{}}Material\\ (\color[HTML]{674EA7}pero\color[HTML]{000000}/\color[HTML]{E69138}PCM\color[HTML]{000000}/\\ \color[HTML]{3D85C6}TMDC\color[HTML]{000000}/\color[HTML]{6AA84F}UCNP\color[HTML]{000000}/\\ \color[HTML]{C27BA0}MSh\color[HTML]{000000})\end{tabular}}}} & \multicolumn{1}{l|}{\begin{tabular}[c]{@{}l@{}}Wavelength \\ range\end{tabular}} & \multicolumn{1}{l|}{\begin{tabular}[c]{@{}l@{}}Tuning \\ method\end{tabular}}                        & \multicolumn{1}{l|}{{\color[HTML]{222222} \begin{tabular}[c]{@{}l@{}}Estimated \\ switching \\ time\end{tabular}}} & refs.                        \\ \hline
\multicolumn{1}{|l|}{\begin{tabular}[c]{@{}l@{}}chemically\\  tunable \\ meta-hologram\end{tabular}}                                    & \multicolumn{1}{l|}{{\color[HTML]{674EA7} \textbf{MAPbX$_3$}}}                                                                            & \multicolumn{1}{l|}{632 nm}                                                      & \multicolumn{1}{l|}{anion exchange}                                                                  & \multicolumn{1}{l|}{{\color[HTML]{222222} 25 min}}                                                                 & \citenum{zhang2019lead}                \\ \hline
\multicolumn{1}{|l|}{\begin{tabular}[c]{@{}l@{}}chemically\\  tunable \\ light-emitting\\  metasurface\end{tabular}}                    & \multicolumn{1}{l|}{{\color[HTML]{674EA7} \textbf{MAPbX$_3$}}}                                                                            & \multicolumn{1}{l|}{500-650 nm}                                                  & \multicolumn{1}{l|}{anion exchange}                                                                  & \multicolumn{1}{l|}{{\color[HTML]{222222} few hours}}                                                              & \citenum{gao2018lead}                  \\ \hline
\multicolumn{1}{|l|}{\begin{tabular}[c]{@{}l@{}}chemically \\ tunable \\ Fano-resonant \\ nanoantennas\end{tabular}}                    & \multicolumn{1}{l|}{{\color[HTML]{674EA7} \textbf{CsPbX$_3$}}}                                                                            & \multicolumn{1}{l|}{410-530 nm}                                                  & \multicolumn{1}{l|}{anion exchange}                                                                  & \multicolumn{1}{l|}{{\color[HTML]{222222} 10 s - 5 min}}                                                           & \citenum{tiguntseva2018tunable}        \\ \hline
\multicolumn{1}{|l|}{\begin{tabular}[c]{@{}l@{}}optically \\ modulated \\ Fano-resonant \\ nanoantennas\end{tabular}}                   & \multicolumn{1}{l|}{{\color[HTML]{674EA7} \textbf{CsPbBr$_3$}}}                                                                           & \multicolumn{1}{l|}{520-540 nm}                                                  & \multicolumn{1}{l|}{\begin{tabular}[c]{@{}l@{}}laser pulse free \\ carriers generation\end{tabular}} & \multicolumn{1}{l|}{{\color[HTML]{222222} 0.5 ps}}                                                                 & \citenum{franceschini2020tuning}       \\ \hline
\multicolumn{1}{|l|}{\begin{tabular}[c]{@{}l@{}}optically \\ modulated \\ THz metasurface\end{tabular}}                                 & \multicolumn{1}{l|}{{\color[HTML]{674EA7} \textbf{\begin{tabular}[c]{@{}l@{}}Ruddlesden-\\ Popper \\ 2D perovskites\end{tabular}}}}    & \multicolumn{1}{l|}{$\sim$1 THz}                                                 & \multicolumn{1}{l|}{\begin{tabular}[c]{@{}l@{}}laser pulse free \\ carriers generation\end{tabular}} & \multicolumn{1}{l|}{{\color[HTML]{222222} 20 ps}}                                                                  & \citenum{kumar2020excitons}            \\ \hline
\multicolumn{1}{|l|}{{\color[HTML]{222222} \begin{tabular}[c]{@{}l@{}}thermally\\ tunable \\ metasurface lens\end{tabular}}}            & \multicolumn{1}{l|}{{\color[HTML]{E69138} \textbf{Ge$_2$Sb$_2$Se$_4$Te}}}                                               & \multicolumn{1}{l|}{5.2 um}                                                      & \multicolumn{1}{l|}{furnace annealing}                                                               & \multicolumn{1}{l|}{{\color[HTML]{222222} 1-10 of mins}}                                                           & \citenum{shalaginov2021reconfigurable} \\ \hline
\multicolumn{1}{|l|}{\begin{tabular}[c]{@{}l@{}}optically \\ tunable \\ metasurface\\ for beam steering\end{tabular}}                   & \multicolumn{1}{l|}{{\color[HTML]{E69138} \textbf{Ge$_2$Sb$_2$Te$_5$}}}                                                 & \multicolumn{1}{l|}{1.5 um}                                                      & \multicolumn{1}{l|}{laser pulse heating}                                                             & \multicolumn{1}{l|}{{\color[HTML]{222222} 10 ns - 1 us}}                                                           & \citenum{galarreta2018nonvolatile}     \\ \hline
\multicolumn{1}{|l|}{\begin{tabular}[c]{@{}l@{}}optically \\ tunable \\ metasurface \\ for structural \\ color generation\end{tabular}} & \multicolumn{1}{l|}{{\color[HTML]{E69138} \textbf{Sb2S3}}}                                                     & \multicolumn{1}{l|}{500-800 nm}                                                  & \multicolumn{1}{l|}{laser pulse heating}                                                             & \multicolumn{1}{l|}{{\color[HTML]{222222} $\sim$100 ms}}                                                           & \citenum{lu2021reversible}             \\ \hline
\multicolumn{1}{|l|}{\begin{tabular}[c]{@{}l@{}}optically \\ controlled\\ waveguide\\  modulators\end{tabular}}                         & \multicolumn{1}{l|}{{\color[HTML]{E69138} \textbf{\begin{tabular}[c]{@{}l@{}}Sb$_2$S$_3$,\\ Sb$_2$Se$_3$\end{tabular}}}}   & \multicolumn{1}{l|}{1550 nm}                                                     & \multicolumn{1}{l|}{laser pulse heating}                                                             & \multicolumn{1}{l|}{{\color[HTML]{222222} 400 ns - 100 ms}}                                                        & \citenum{delaney2020new}               \\ \hline
\multicolumn{1}{|l|}{\begin{tabular}[c]{@{}l@{}}optically \\ tunable \\ Huygens \\ metasurface\end{tabular}}                            & \multicolumn{1}{l|}{{\color[HTML]{E69138} \textbf{Ge/Ge$_3$Sb$_2$Te$_6$}}}                                              & \multicolumn{1}{l|}{3.8 um}                                                      & \multicolumn{1}{l|}{laser pulse heating}                                                             & \multicolumn{1}{l|}{{\color[HTML]{222222} 10 ns - 1 us}}                                                           & \citenum{leitis2020all}                \\ \hline
\end{tabular}
\end{table}

\begin{table}[]
\begin{tabular}{|l|l|l|l|l|l|}
\hline
\begin{tabular}[c]{@{}l@{}}optically\\  tunable \\ metasurface \\ for multilevel \\ spectral filtering\end{tabular} & {\color[HTML]{E69138} \textbf{Si/Ge$_2$Sb$_2$Te$_5$}}                                                    & 1.3-1.5 um & \begin{tabular}[c]{@{}l@{}}laser pulse\\  heating \end{tabular}   & {\color[HTML]{222222} 10 ns - 1 us} &\citenum{galarreta2020reconfigurable}                                                          \\ \hline
\begin{tabular}[c]{@{}l@{}}electrically \\ tunable \\ metasurfaces\end{tabular}                                     & {\color[HTML]{E69138} \textbf{Ge$_2$Sb$_2$Te$_5$}}                                                       & 0.7-1.5 um & electrical current    & {\color[HTML]{222222} 500 ns - 5us} & \citenum{wang2021electrical, zhang2021electrically}  \\ \hline
\begin{tabular}[c]{@{}l@{}}optically \\ controlled\\ metasurface \\ light modulator\end{tabular}                    & {\color[HTML]{3D85C6} \textbf{MoSe$_2$}}                                                                                   & 750 nm     & laser pulses          & 10-100 ps                           & \citenum{kravtsov2020nonlinear}                                                                \\ \hline
\begin{tabular}[c]{@{}l@{}}electrically\\ tunable\\ frequency\\ conversion\end{tabular}                             & {\color[HTML]{3D85C6} \textbf{WSe$_2$}}                                                                                    & 705 nm     & gate voltage          & undefined                           & \citenum{seyler2015electrical}                                                                 \\ \hline

\begin{tabular}[c]{@{}l@{}}electrically\\ switchable\\ Fresnel lens\end{tabular}                             & {\color[HTML]{3D85C6} \textbf{WS$_2$}}                                                                                    & 620 nm     & gate voltage          & 40 ms                           & \citenum{van2020exciton}                                                                \\ \hline



\begin{tabular}[c]{@{}l@{}}temperature\\ tunablelight\\ emission\end{tabular}                                       & {\color[HTML]{6AA84F} \textbf{\begin{tabular}[c]{@{}l@{}}NaYF$_4$/Tm$^{3+}$ NPs \\ on polystyrene\\  microsphere\end{tabular}}} & 805 nm     & \begin{tabular}[c]{@{}l@{}}light-induced\\  heating \end{tabular}  & undefined                           & \citenum{fernandez2018continuous}                                                              \\ \hline
\begin{tabular}[c]{@{}l@{}}optically\\ tunable \\ light scattering\end{tabular}                                     & {\color[HTML]{C27BA0} \textbf{Si/PSS}}                                                                                  & 600-650 nm & \begin{tabular}[c]{@{}l@{}}light-induced\\  heating \end{tabular} & 20 min                              & \citenum{nikitina2021all}                                                                      \\ \hline
\multicolumn{6}{|c|}{\textbf{Advanced laser architectures}}                                                                                                                                                                                                                                                                                                                                                                                                                                                                                                                                                                                                        \\ \hline
\multicolumn{1}{|l|}{Application}                                                                                                       & \multicolumn{1}{l|}{{ \textbf{\begin{tabular}[c]{@{}l@{}}Material\\ (\color[HTML]{674EA7}pero\color[HTML]{000000}/\color[HTML]{E69138}PCM\color[HTML]{000000}/\\ \color[HTML]{3D85C6}TMDC\color[HTML]{000000}/\color[HTML]{6AA84F}UCNP\color[HTML]{000000}/\\ \color[HTML]{C27BA0}MSh\color[HTML]{000000})\end{tabular}}}}                               & \multicolumn{1}{l|}{\begin{tabular}[c]{@{}l@{}}Wavelength \\ range\end{tabular}}     & \multicolumn{1}{l|}{\begin{tabular}[c]{@{}l@{}}Operation \\ temperature\end{tabular}} & \multicolumn{1}{l|}{{\color[HTML]{222222} \begin{tabular}[c]{@{}l@{}}Pump type \&\\ threshold power\end{tabular}}} & refs.                                           \\ \hline
\multicolumn{1}{|l|}{\begin{tabular}[c]{@{}l@{}}self-resonant \\ Mie nanolaser\end{tabular}}                                            & \multicolumn{1}{l|}{{\color[HTML]{674EA7} \textbf{CsPbBr$_3$}}}                                                                                                         & \multicolumn{1}{l|}{535 nm}                                                          & \multicolumn{1}{l|}{room}                                                             & \multicolumn{1}{l|}{{\color[HTML]{222222} \begin{tabular}[c]{@{}l@{}}fs pulsed\\ 70 $\mu$J/cm$^2$\end{tabular}}}          & \citenum{tiguntseva2020room}      \\ \hline
\multicolumn{1}{|l|}{\begin{tabular}[c]{@{}l@{}}nanowaveguide-\\ integrated lasers\end{tabular}}                                        & \multicolumn{1}{l|}{{\color[HTML]{674EA7} \textbf{(Cs,MA)PbX$_3$}}}                                                                                                     & \multicolumn{1}{l|}{400-800 nm}                                                      & \multicolumn{1}{l|}{room}                                                             & \multicolumn{1}{l|}{{\color[HTML]{222222} \begin{tabular}[c]{@{}l@{}}fs pulsed\\ 0.2 - 20 $\mu$J/cm$^2$\end{tabular}}}    & \citenum{zhang2019controlled}     \\ \hline
\multicolumn{1}{|l|}{\begin{tabular}[c]{@{}l@{}}vortex\\ metasurface-\\ based laser\end{tabular}}                                         & \multicolumn{1}{l|}{{\color[HTML]{674EA7} \textbf{MAPbBr$_3$}}}                                                                                                         & \multicolumn{1}{l|}{552 nm}                                                          & \multicolumn{1}{l|}{room}                                                             & \multicolumn{1}{l|}{{\color[HTML]{222222} \begin{tabular}[c]{@{}l@{}}fs pulsed \\ 4.2 $\mu$J/cm$^2$\end{tabular}}}        & \citenum{huang2020ultrafast}      \\ \hline
\multicolumn{1}{|l|}{\begin{tabular}[c]{@{}l@{}}interlayer \\ exciton laser\end{tabular}}                                               & \multicolumn{1}{l|}{{\color[HTML]{3D85C6} \textbf{WSe$_2$/MoSe$_2$}}}                                                                                                      & \multicolumn{1}{l|}{905 nm}                                                          & \multicolumn{1}{l|}{5K}                                                               & \multicolumn{1}{l|}{{\color[HTML]{222222} \begin{tabular}[c]{@{}l@{}}fs pulsed\\ 6.2 $\mu$J/cm$^2$\end{tabular}}}         & \citenum{paik2019interlayer}      \\ \hline
\multicolumn{1}{|l|}{microflake laser}                                                                                                  & \multicolumn{1}{l|}{{\color[HTML]{3D85C6} \textbf{InSe}}}                                                                                                            & \multicolumn{1}{l|}{1030 nm}                                                         & \multicolumn{1}{l|}{room}                                                             & \multicolumn{1}{l|}{{\color[HTML]{222222} \begin{tabular}[c]{@{}l@{}}fs pulsed\\ 0.62 mJ/cm$^2$\end{tabular}}}        & \citenum{li2022room}              \\ \hline
\multicolumn{1}{|l|}{{\color[HTML]{222222} \begin{tabular}[c]{@{}l@{}}photonic crystal \\ cavity laser\end{tabular}}}                   & \multicolumn{1}{l|}{{\color[HTML]{3D85C6} \textbf{WSe$_2$}}}                                                                                    & \multicolumn{1}{l|}{740 nm}                                                          & \multicolumn{1}{l|}{80K}                                                              & \multicolumn{1}{l|}{{\color[HTML]{222222} \begin{tabular}[c]{@{}l@{}}CW\\ 1 W/cm$^2$\end{tabular}}}                   & \citenum{wu2015monolayer}         \\ \hline
\multicolumn{1}{|l|}{\begin{tabular}[c]{@{}l@{}}multicolor \\ upconversion \\laser\end{tabular}}                                          & \multicolumn{1}{l|}{{\color[HTML]{6AA84F} \textbf{\begin{tabular}[c]{@{}l@{}}NaYF$_4$/Tm$^{3+}$ NPs \\ on polystyrene \\ microsphere\end{tabular}}}} & \multicolumn{1}{l|}{\begin{tabular}[c]{@{}l@{}}440-520 nm\\ 780-840 nm\end{tabular}} & \multicolumn{1}{l|}{room}                                                             & \multicolumn{1}{l|}{{\color[HTML]{222222} \begin{tabular}[c]{@{}l@{}}CW\\ 14 kW/cm$^2$\end{tabular}}}                 & \citenum{fernandez2018continuous} \\ \hline


\end{tabular}
\end{table}





\begin{table}[h!]
\begin{tabular}{|llllll|}
\hline
\multicolumn{6}{|c|}{\textbf{Light frequency conversion optical metadevices}}                                                                                                                                                                                                                                                                                                                                                                                                                                                                                                                                      \\ \hline
\multicolumn{1}{|l|}{Application}                                                                                              & \multicolumn{1}{l|}{{ \textbf{\begin{tabular}[c]{@{}l@{}}Material\\ (\color[HTML]{674EA7}pero\color[HTML]{000000}/\color[HTML]{E69138}PCM\color[HTML]{000000}/\\ \color[HTML]{3D85C6}TMDC\color[HTML]{000000}/\color[HTML]{6AA84F}UCNP\color[HTML]{000000}/\\ \color[HTML]{C27BA0}MSh\color[HTML]{000000})\end{tabular}}}} & \multicolumn{1}{l|}{\begin{tabular}[c]{@{}l@{}}Wavelength\\  range,\\ pump/\\ emission\end{tabular}} & \multicolumn{1}{l|}{nonlinear process}                                                        & \multicolumn{1}{l|}{{\color[HTML]{222222} \begin{tabular}[c]{@{}l@{}}External \\ efficiency\end{tabular}}} & refs.                \\ \hline
\multicolumn{1}{|l|}{\begin{tabular}[c]{@{}l@{}}IR converters\\ based on \\ metasurface\end{tabular}}                          & \multicolumn{1}{l|}{{\color[HTML]{674EA7} \textbf{MAPbBr$_3$}}}                                                                           & \multicolumn{1}{l|}{\begin{tabular}[c]{@{}l@{}}650 nm/\\ 532 nm\end{tabular}}                        & \multicolumn{1}{l|}{\begin{tabular}[c]{@{}l@{}}multiphoton \\ photoluminescence\end{tabular}} & \multicolumn{1}{l|}{{\color[HTML]{222222} 2\%}}                                                            & \citenum{fan2021enhanced}      \\ \hline
\multicolumn{1}{|l|}{\begin{tabular}[c]{@{}l@{}}dielectric \\ metasurface \\ enhanced \\ frequency \\ conversion\end{tabular}} & \multicolumn{1}{l|}{{\color[HTML]{0070C0} \textbf{MoS$_2$}}}                                                      & \multicolumn{1}{l|}{\begin{tabular}[c]{@{}l@{}}810-850 nm/ \\ 405-425 nm\end{tabular}}               & \multicolumn{1}{l|}{\begin{tabular}[c]{@{}l@{}}second harmonic \\ generation\end{tabular}}    & \multicolumn{1}{l|}{{\color[HTML]{222222} 4.5×10−7 \%}}                                                    & \citenum{loechner2020hybrid}   \\ \hline
\multicolumn{1}{|l|}{\begin{tabular}[c]{@{}l@{}}switchable \\ IR converters \\ based on \\ nanoparticles\end{tabular}}         & \multicolumn{1}{l|}{{\color[HTML]{FF9900} \textbf{Ge$_2$Sb$_2$Te$_5$}}}                                                                         & \multicolumn{1}{l|}{\begin{tabular}[c]{@{}l@{}}1050 nm/ \\ 525 nm\end{tabular}}                      & \multicolumn{1}{l|}{\begin{tabular}[c]{@{}l@{}}second harmonic \\ generation\end{tabular}}    & \multicolumn{1}{l|}{{\color[HTML]{222222} undefined}}                                                      & \citenum{rybin2021optically}   \\ \hline
\multicolumn{1}{|l|}{\begin{tabular}[c]{@{}l@{}}dual-resonance \\ all-dielectric \\ metasurface \\ with UCNPs\end{tabular}}    & \multicolumn{1}{l|}{{\color[HTML]{6AA84F} \textbf{\begin{tabular}[c]{@{}l@{}}NaYF$_4$:Yb/Tm NPs\\ on Si metasurface\end{tabular}}}}       & \multicolumn{1}{l|}{\begin{tabular}[c]{@{}l@{}}965 nm/ \\ 800 nm\end{tabular}}                       & \multicolumn{1}{l|}{\begin{tabular}[c]{@{}l@{}}upconversion \\ fluorescence\end{tabular}}     & \multicolumn{1}{l|}{{\color[HTML]{222222} 3-4\%}}                                                          & \citenum{gong2019upconversion} \\ \hline
\multicolumn{1}{|l|}{\begin{tabular}[c]{@{}l@{}}frequency \\ conversion \\ in colloidal \\ meta-shells\end{tabular}}           & \multicolumn{1}{l|}{{\color[HTML]{C27BA0} \textbf{polystyrene/ZnO}}}                                                                   & \multicolumn{1}{l|}{\begin{tabular}[c]{@{}l@{}}900 nm/\\ 450 nm\end{tabular}}                        & \multicolumn{1}{l|}{\begin{tabular}[c]{@{}l@{}}second harmonic \\ generation\end{tabular}}    & \multicolumn{1}{l|}{{\color[HTML]{222222} 10-5 \%}}                                                        & \citenum{bahng2020mie}         \\ \hline
\end{tabular}
\end{table}


Below we outline several directions where we may expect a rapid progress in the coming years.

{\it Towards hybrid metaphotonics.} Each of the materials discussed above possesses unique properties which are complementary in many cases. Thus, mutual integration of different materials can bring novel functionalities for metaphotonics. The simplest approach is just to mix different materials in a solution. For example, mixing UCNPs with halide perovskite layers was shown to be prospective for photovoltaics, where the upconversion helps to harvest additionally the IR part of the sun spectrum, boosting the energy conversion of perovskite solar cells~\cite{he2016monodisperse} or photodetectors~\cite{zhang2017perovskite}. Further improvement of the device performance with such nanocomposites can be done by their integration with metastructures possessing high-$Q$ optical resonances at the IR frequencies. 

Another approach is to employ coating of some functional cores (e.g. UCNP or perovskite nanolaser) by a metallic shell, which can modify strongly  core's optical resonances or/and add novel functionalities. In connection to this strategy employing core-shell structural elements, it is worth mentioning many achievements in creating reversed metal-dielectric nanoparticles basically driven by plasmon resonances~\cite{volk,magnozzi} instead of the Mie resonances explored in metaphotonics. 

One more approach for the materials integration is to combine nanomaterials obtained from a liquid phase with solid materials. In other words, such materials as halide perovskites can be grown chemically on PCMs or TMDCs directly forming heterostructures~\cite{voznyy2017engineering}, which are interesting for optically switchable lasers, multi-level data recording, and storage. Optimized materials and nanophotonic designs can also be useful for local temperature control~\cite{zograf2021all}.

One of the promising strategies may be the use of amorphous metal oxide nanosheets for tunability of metamaterial structures. Recently, universal methodology was demonstrated for fabricating two-dimensional amorphous metal oxides  that exhibit perfect electrochemical properties and long-term cyclic stability~\cite{amo}.  It is believed that two-dimensional amorphous nanomaterials will find more applications in catalysis and energy storage, and their properties can be enhanced by employing the concepts of metaphotonics. 

{\it Enhancing light-matter interaction.} One of the important future general directions in metaphotonics is to explore deeper new ways to control coupling of light with matter~\textit{via} polaritonics. Polaritonic structures emerged as a new paradigm that drives light–matter interaction at the nanoscale in many systems. Polaritons allow to couple electromagnetic waves with dipole-active matter excitations such as plasmonic electron oscillations in metals, excitonic electron-hole pairs in semiconductors, or phonon lattice vibrations, where high-$Q$ resonances provided by metasystems may allow to control and enhance multiple effects. With the emergence of 2D materials, many of such effects can be explored and manipulated in flatland, so that the engineered metasurfaces can provide a useful interface for light-emitting quantum systems or serve as light sources themselves. The fundamental theory, advanced nanofabrication, new approaches for synthesis of low-dimensional materials, and atomic-scale material characterization are expected to drive the development of new interdisciplinary paradigms based on chemical physics, quantum optics,  and polaritonic systems. 

{\it Developing closer links to chemistry.} In a broader perspective, metaphotonics may underpin the development of new multifunctional nanoscale systems for efficient photocatalysis based on perovskite oxides and functionalized nanocarbon matrices. The crystal structure of perovskites provides an excellent basis for adjusting the band gap values, which ensure absorption of visible light and edge band potentials in accordance with the needs for specific photocatalytic reactions. An additional component for such metasystems that expands their applications to photocatalysis are nanocarbon matrices as well as their derivatives which may play an important role in an increase of the active surface area with precise control of the type and concentration of the catalyst. Carbon nanoscale materials can be functionalized using chiral molecules such as aspartic acid and cysteine for efficient enantioselective photocatalysis, providing a great advantage in the formation of multifunctional metasystems with unique properties. Remarkably, chiral plasmonic nanoparticles synthesized with bottom-up solution platforms can additionally enhance such effects~\cite{lee2018amino}.

Photocatalysis can induce effectively chemical reactions by light exposure at room temperature for novel technologies that can convert clean, safe, and abundant solar energy into electrical or chemical energy.  New breakthrough technologies in this area should include production of H$_2$ and O$_2$ from water~\cite{nishiyama_2021}, reduction of CO$_2$, and an increase of the yield of asymmetric chemical reactions. Novel nanostructured materials with engineered high-$Q$ resonances can be useful for efficient photocatalytic reactions solving important problems of green energy technologies. More specifically, patterned metasurfaces can be employed for artificial photosynthesis to enhance activity and selectivity of the two-electron oxygen reduction reaction in the photocatalytic process~\cite{teng_2021} enhancing the production efficiency of hydrogen peroxide (H$_2$O$_2$)—an environmentally friendly oxidant and a clean fuel. 


\newpage

\section*{Authors' biographies}

{\bf Pavel Tonkaev} received his MS degree in technical physics in 2019 at the ITMO University (St. Petersburg, Russia). Currently, he is a PhD student at the Australian National University, and his research interests are in the fields of nanophotonics, nanolasers, and nonlinear photoluminescence.

{\bf Ivan Sinev} received his MS degree (2014) at St. Petersburg Academic University and his PhD. degree (2018) at ITMO University. Currently, he is an assistant professor at the School of Physics and Engineering at ITMO University. His research interests include tunable metasurfaces and nanoantennas, including those based on phase change materials, and light-matter interaction in 2D materials coupled to photonic structures.

{\bf Mikhail Rybin} received PhD (2009) and Habilitation (Doctor of Science) (2018) degrees in physics from the Ioffe Institute (St. Petersburg, Russia).  In 2010, he joined the School of Physics and Engineering at ITMO University, where currently he is a senior research fellow. The current topics of his research include phase change materials, quasicrystals, resonant interaction of light with photonic structures, Fano resonances, all-dielectric metamaterials, photonic crystals and nanoantennas.

{\bf Sergey Makarov} is Professor, Head of Laboratory of Hybrid Nanophotonics and Optoelectronics, Dean of Faculty of Photonics, as well as Director of Shared Research Facilities on Nanotechnology at the ITMO University. He received a PhD degree in 2014 from the Lebedev Physics Institute of the Russian Academy of Sciences (Moscow), and Habilitation (DSc) from the ITMO University (St.~Petersburg, Russia). The topics of his research activity include halide perovskites, nanophotonics, laser-matter interaction, and nanotechnology.

{\bf Yuri Kivshar} is Distinguished Professor of the Australian National University working in photonics and metamaterials. He received PhD degree in 1984 in Kharkov (Ukraine), and in 1989 he
left the Soviet Union to work in Spain and then later in Germany. In 1993, he moved to Australia where he established Nonlinear Physics Center. He is one of the pioneers of the new field of all-dielectric resonant metaphotonics (or "Mie-tronics"), governed by the physics of Mie resonances. He is Fellow of the Australian Academy of Science, OSA, APS, SPIE, and IOP, and the recipient of 2022 Max Born Award from Optica (former OSA). 

\begin{acknowledgement}

This work was supported by the Australian Research Council (grants DP200101168 and DP210101292), and the Ministry of Science and Higher Education of the Russian Federation (project 075-15-2021-589). P.T. acknowledges a support from the RPMA grant of the School of Physics and Engineering of the ITMO University. I.S. acknowledges a support from the Russian Federation President Grant Council, project MK-4652.2021.1.2. This research was supported by Priority 2030 Federal Academic Leadership Program. The authors thank Zongyou Yin and Sofia Morozova for useful suggestions. Y.K. thanks Prof. Tery Odom for her kind invitation to contribute this review paper into a special series on metamaterials. 

\end{acknowledgement}

\newpage 

\section*{Abbreviations}
0D: zero-dimensional

1D: one-dimensional

2D: two-dimensional

3D: three-dimensional

AR: augmented reality

BIC: bound state in continuum

CSU: cooperative sensitization upconversion

CVD: chemical vapor deposition 

CW: continuous-wave

ESA: excited state absorption

ETU: energy transfer upconversion 

GST: Ge$_2$Sb$_2$Te$_5$

IMT: insulator-metal transition

IR: infrared

LED: light-emitting diode

MBE: molecular beam epitaxy 

MSP: meta-shell supraparticle 

NC: nanocrystal

NP: nanoparticle

PA: photon avalanche

PCM: phase change material

PDADMAC: poly-(diallyldimethylammonium chloride)

PL: photoluminescence 

PSS: poly-(sodium 4-styrenesulfonate)

SEM: scanning electron microscopy

SHG: second harmonic generation

STEM: scanning transmission electron microscopy

TMDC: transition metal dichalcogenid

UCNP: upconversion nanoparticle

UV: ultraviolet

VR: virtual reality

XRD: X-ray diffraction 


\newpage 

%%%%%%%%%%%%%%%%%%%%%%%%%%%%%%%%%%%%%%%%%%%%%%%%%%%%%%%%%%%%%%%%%%%%%
%% The same is true for Supporting Information, which should use the
%% suppinfo environment.
%%%%%%%%%%%%%%%%%%%%%%%%%%%%%%%%%%%%%%%%%%%%%%%%%%%%%%%%%%%%%%%%%%%%%
%%\begin{suppinfo}

%%This will usually read something like: ``Experimental procedures and
%%characterization data for all new compounds. The class will
%%automatically add a sentence pointing to the information on-line:

%%\end{suppinfo}

%%%%%%%%%%%%%%%%%%%%%%%%%%%%%%%%%%%%%%%%%%%%%%%%%%%%%%%%%%%%%%%%%%%%%
%% The appropriate \bibliography command should be placed here.
%% Notice that the class file automatically sets \bibliographystyle
%% and also names the section correctly.
%%%%%%%%%%%%%%%%%%%%%%%%%%%%%%%%%%%%%%%%%%%%%%%%%%%%%%%%%%%%%%%%%%%%%

%\bibliographystyle{achemso}
\bibliography{achemso_ref}

\end{document}
